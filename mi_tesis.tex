\textbf{Agencia oscura en entornos de baja institucionalidad: innovación
intraemprendedora y transgresión normativa en organizaciones de
servicios}\\
\emph{Dark Agency in Institutional Voids: Intrapreneurial Innovation and
Bureaucratic Rule-Breaking in Service Organizations}

\emph{James}\footnote{\url{https://orcid.org/0009-0006-2690-2207}}

\textbf{Resumen}

\begin{quote}
\textbf{Propósito:} La literatura sobre el ``lado oscuro'' del
emprendimiento tiende a confundir la malevolencia con la ambición
estratégica. Este estudio examina si el componente residual de agencia
oscura, una vez separado del Factor General Antagónico (\(G\)) dentro de
la Tétrada Oscura, se asocia con el comportamiento intraemprendedor
(EIB) y la transgresión normativa instrumental en un contexto de vacíos
institucionales.

\textbf{Diseño/metodología:} Se emplea un diseño transversal con una
muestra proyectada de \(N \geq 500\) trabajadores del sector servicios
en Perú. Se estima un modelo de medida Bifactor S-1 sobre la escala SD4
para aislar la varianza atribuible a \(G\) y al factor específico de
agencia oscura (\(S_{Agencia}\)). Posteriormente, se analizará un modelo
de mediación moderada, interpretado como configuración disposicional
concurrente, en el que la vigilancia estratégica del entorno actúa como
mecanismo explicativo, condicionado por la percepción de política
organizacional (POPS) y el capital psicológico (PsyCap).

\textbf{Hallazgos esperados:} Se prevé que \(S_{Agencia}\) muestre una
asociación incremental positiva con el comportamiento intraemprendedor y
con las conductas contraproducentes organizacionales de tipo burocrático
(CWB-O), mientras que \(G\) se asociará principalmente con conductas
contraproducentes interpersonales (CWB-I) y con menores niveles de
intraemprendimiento (Paulhus \& Williams, 2002; Moshagen et al., 2018;
Pletzer, 2021). Asimismo, se anticipa que la vigilancia estratégica del
entorno medie estas asociaciones, en términos de configuración
disposicional concurrente, y que la POPS y el PsyCap faciliten, bajo
ciertas condiciones, la expresión funcional de la agencia oscura
(Hochwarter et al., 2003; Tang, 2020; Berisha et al., 2025;
Gojny-Zbierowska, 2024).

\textbf{Originalidad/valor:} El estudio aborda explícitamente el
problema de ``contaminación de varianza'' en la investigación de rasgos
oscuros (Moshagen et al., 2018; Zettler et al., 2021) y propone el
perfil de ``desviación constructiva orientada al logro'', sugiriendo que
determinados rasgos aversivos, depurados de su núcleo abiertamente
antagónico, pueden ser funcionales para la innovación en entornos
informales y burocráticos, aunque sigan siendo ambivalentes en términos
normativos (Malisetty \& Kumari, 2018; Jordan et al., 2021).

\textbf{Abstract}

\textbf{Purpose:} Research on the ``dark side'' of entrepreneurship
often conflates malevolence with strategic ambition. This study examines
whether the residual Dark Agency component (\(S_{Agency}\)), once
separated from the General Antagonistic Factor (\(G\)) within the Dark
Tetrad, is associated with employee intrapreneurial behavior (EIB) and
instrumental norm-breaking in a context of institutional voids (Baumol,
1990; Webb et al., 2013).

\textbf{Design/Methodology:} A cross-sectional design will be employed
with \(N \geq 500\) service-sector employees in Peru. A Bifactor S-1
measurement model will be estimated on the SD4 scale to isolate the
variance of \(G\) and \(S_{Agency}\) (Eid et al., 2017; Paulhus et al.,
2021). A moderated mediation model, interpreted as a dispositional
concurrent configuration, will be tested in which Strategic
Environmental Scanning functions as the explanatory mechanism,
conditioned by Perceived Organizational Politics and Psychological
Capital (Ashford \& Black, 1996; Tang et al., 2012; Hochwarter et al.,
2003; Luthans et al., 2007).

\textbf{Expected Findings:} \(S_{Agency}\) is expected to show a
positive incremental association with EIB and organizational
counterproductive work behaviors (CWB-O), whereas G will predominantly
relate to interpersonal counterproductive work behaviours (CWB-I) and
lower levels of EIB (Moshagen et al., 2018; Zettler et al., 2021;
Pletzer, 2021). Strategic environmental scanning is also expected to
mediate this pattern, being activated by the perception of a politicized
environment, while POPS and PsyCap may help channel dark agency toward
achievement-oriented constructive deviance rather than purely
destructive outcomes (Dang-Van et al., 2022; Tang, 2020; Berisha et al.,
2025).

\textbf{Originality/Value:} The study addresses ``variance
contamination'' issues in dark trait research (Moshagen et al., 2018)
and advances the profile of achievement-oriented constructive deviance,
suggesting that certain aversive traits, once their general antagonistic
core is partialled out, may be functionally adaptive for innovation in
informal and bureaucratic settings (Malisetty \& Kumari, 2018; Jordan et
al., 2021).
\end{quote}

\textbf{CAPÍTULO 1. PLANTEAMIENTO DEL PROBLEMA}

\textbf{1.1. Vacíos institucionales, informalidad y naturaleza
discrecional de la innovación}

La teoría clásica de la asignación del emprendimiento sostiene que el
talento emprendedor no es intrínsecamente ``bueno'' o ``malo'', sino que
su orientación productiva, improductiva o abiertamente destructiva
depende de la estructura de incentivos que proveen las instituciones de
una economía (Baumol, 1990). En contextos con instituciones formales
relativamente sólidas, los marcos regulatorios, los sistemas judiciales
y los mecanismos de protección de derechos de propiedad tienden a
alinear la actividad innovadora con resultados productivos y socialmente
deseables, porque reducen la rentabilidad esperada de estrategias
depredadoras y de captura de rentas puramente especulativas.

En contraste, en economías emergentes caracterizadas por vacíos
institucionales ---regulación inefectiva, burocracia opaca, altos
niveles de informalidad y cumplimiento selectivo de las normas--- la
relación entre emprendimiento e innovación es menos lineal (Webb et al.,
2013; Welter \& Smallbone, 2011). La literatura latinoamericana de
economía política ha insistido en que estas configuraciones no son
anomalías coyunturales, sino expresiones persistentes de heterogeneidad
estructural: coexistencia de segmentos modernos, altamente integrados a
mercados globales, con amplias franjas de empleo precario, baja
productividad y débil protección institucional (Pinto, 1970; Jiménez,
2010). En el caso peruano, la estructura productiva se ha descrito como
una economía ``dual y fragmentada'', donde una parte del aparato moderno
convive con un vasto ``mundo del trabajo marginal'' en el que predominan
relaciones laborales informales, ingresos inestables y escasa cobertura
de derechos (Cavero, 2017; Jiménez, 2010).

En estos entornos, la existencia de ``zonas grises'' normativas facilita
la emergencia de estrategias empresariales que combinan creación de
valor con prácticas que eluden, reinterpretan o tensan los marcos
formales, especialmente en segmentos intensivos en servicios donde la
supervisión es más ambigua y la presión por resultados es inmediata
(Webb et al., 2013). Como señalan los análisis de heterogeneidad
productiva aplicados al Perú, la informalidad no es simplemente una
``falla'' a corregir, sino un modo de funcionamiento estable que
articula mercados de trabajo, estructuras empresariales y políticas
públicas de manera desigual, generando circuitos donde la frontera entre
lo formal y lo informal es porosa y negociada (Cavero, 2017; Jiménez,
2010).

La literatura sobre informalidad muestra que, cuando los marcos formales
son costosos, poco confiables o percibidos como injustos, los actores
recurren a arreglos informales para reducir incertidumbre y capturar
oportunidades (Webb et al., 2009; Webb et al., 2013; Welter \&
Smallbone, 2011). Desde una perspectiva más reciente, estos entornos se
entienden menos como ``vacíos'' y más como contextos de alta complejidad
institucional, donde coexisten demandas normativas conflictivas y opacas
(Luo, 2022). En el caso peruano, esta complejidad se expresa en la
superposición de reglas formales, prácticas administrativas
discrecionales y normas informales localmente arraigadas, en un contexto
de elevada desigualdad y persistente exclusión social (Jiménez, 2010;
Cavero, 2017).

Psicológicamente, esta complejidad se experimenta como presión cotidiana
para navegar la ambigüedad mediante redes de influencia y astucia
política, de modo que los vacíos institucionales actúan como
catalizadores de estrategias adaptativas que, aunque desviadas en
términos formales, resultan funcionales para reducir la incertidumbre y
capturar oportunidades (Luo, 2022). Desde los lentes de la economía
política, esto equivale a decir que las mismas fallas estructurales que
sostienen la informalidad también generan un campo de juego donde la
capacidad de moverse en esas zonas grises ---leyendo reglas implícitas,
negociando con autoridades, acomodando normas según la coyuntura--- se
convierte en un recurso diferencial de poder y de agencia.

En ese sentido, la Percepción de Política Organizacional (POPS) se
concibe como un correlato psicológico coherente con, y probablemente más
frecuente en, contextos de baja institucionalidad percibida: cuando las
reglas del juego macro son difusas o se aplican de forma desigual, los
individuos tienden a interpretar el entorno organizacional como
altamente politizado, donde las decisiones dependen menos de criterios
meritocráticos y más de coaliciones, lealtades y negociación informal
(Hochwarter et al., 2003; Webb et al., 2013). En este estudio, la baja
institucionalidad se asume como marco contextual interpretativo y su
dimensión subjetiva proximal se operacionaliza empíricamente mediante la
POPS.

Más allá de la argumentación clásica sobre vacíos institucionales, la
evidencia reciente en contextos latinoamericanos sugiere que la
combinación entre baja institucionalidad y reglas ambiguas no solo
afecta la tasa de emprendimiento, sino también el tipo de perfiles
psicológicos que encuentran tracción en ese entorno. Estudios con
muestras latinas han mostrado que los rasgos de la Tríada/Tétrada Oscura
se vinculan con la intención de emprender y con formas específicas de
emprendimiento naciente, indicando que el ``lado oscuro'' puede
traducirse en estrategias de aprovechamiento de asimetrías
institucionales y redes informales más que en destrucción abierta del
orden económico (Afshar Jahanshahi et al., 2025).

En otras palabras, los mismos vacíos institucionales que generan riesgo
y precariedad también amplifican el valor instrumental de la frialdad
estratégica, la tolerancia al conflicto y la disposición a reinterpretar
creativamente las normas, rasgos que ---en este estudio--- se
conceptualizan como expresión de Agencia Oscura en organizaciones de
servicios peruanas. En esta tesis, el ``vacío institucional'' peruano no
se entiende, por tanto, como un simple desorden difuso, sino como un
ecosistema social particular donde la retención estratégica de
información y la venganza diferida funcionan como mecanismos ordinarios
de coordinación y castigo. La evidencia de análisis de redes en muestras
peruanas sugiere que los contenidos vinculados con guardar información
para usarla más adelante y esperar el momento adecuado para vengarse
ocupan posiciones centrales en la arquitectura de la Tétrada Oscura,
indicando que el control del tiempo, del acceso a datos sensibles y de
las represalias simbólicas constituye un recurso clave de poder. En un
entorno así, la Agencia Oscura no opera solo como ``maldad''
idiosincrática, sino como repertorio adaptativo para navegar
instituciones fragmentadas, donde leer correctamente qué regla es letra
muerta, qué norma sí se sanciona y qué alianzas ofrecen protección
política se vuelve una condición de posibilidad para innovar y sostener
proyectos intraemprendedores.

En este escenario, la innovación corporativa deja de ser un resultado
predecible de procesos organizacionales estandarizados y pasa a depender
en mayor medida de la agencia individual discrecional. Mantener
comportamientos intraemprendedores (Employee Intrapreneurship Behavior,
EIB) ---es decir, comportamiento de exploración, generación e
implementación de ideas nuevas dentro de la organización (Gawke et al.,
2019)--- exige perfiles psicológicos capaces de tolerar ambigüedad,
gestionar conflictos y operar estratégicamente en entornos de reglas
inestables.

La literatura tradicional de comportamiento organizacional ha tendido a
privilegiar rasgos ``luminosos'' (e.g., consciencia, apertura,
proactividad) como predictores de innovación y desempeño (Rauch \&
Frese, 2007). Sin embargo, el auge de la investigación sobre la Tétrada
Oscura (narcisismo, maquiavelismo, psicopatía y sadismo) sugiere que
ciertos rasgos con connotaciones aversivas podrían, bajo condiciones
específicas, asociarse también con resultados funcionales, aunque
potencialmente acompañados de costos sociales o normativos (Moshagen et
al., 2018; Paulhus \& Williams, 2002; Zettler et al., 2021). En
contextos de baja institucionalidad, este ``lado oscuro'' podría no ser
solo un problema, sino también una fuente de tracción para sostener
innovación de alto riesgo y alta exposición.

Esta problemática se ha vuelto más evidente en el contexto postpandemia,
donde la inseguridad laboral, la angustia psicológica y la presión por
resultados han incrementado tanto la necesidad de innovar como el riesgo
de conductas desviadas (Dang-Van et al., 2022). La evidencia reciente
sugiere que, en entornos de crisis y recursos restringidos, la
sostenibilidad del esfuerzo intraemprendedor depende en buena medida de
la capacidad de los trabajadores para movilizar recursos psicológicos
positivos, como el PsyCap, que amortigüen el impacto de la incertidumbre
y permitan reconfigurar oportunidades (Tang, 2020; Milosevic et al.,
2017). En este sentido, analizar cómo la Agencia Oscura interactúa con
el PsyCap para inclinar la balanza hacia la desviación constructiva más
que hacia la destrucción normativa ofrece un aporte sustantivo a la
comprensión de la adaptación organizacional en entornos hostiles
(Berisha et al., 2025; Gojny-Zbierowska, 2024).

Esta lectura no pretende romantizar la transgresión ni normalizar la
corrupción, sino explicitar el tipo de riesgo epistémico que se asume al
estudiar el lado oscuro en contextos periféricos. Tratar toda forma de
desviación como patología moral supone el peligro de producir falsos
negativos sistemáticos: pasar por alto configuraciones de rasgos
aversivos que, en entornos de baja institucionalidad, cumplen funciones
organizacionales de supervivencia. Separar empíricamente la malevolencia
general del componente agéntico residual implica, en cambio, aceptar un
riesgo controlado de falsos positivos ---atribuir funcionalidad a
perfiles potencialmente problemáticos--- a cambio de mapear con mayor
precisión cómo se sostienen la innovación y la adaptación en estructuras
institucionales incompletas. La tesis adopta deliberadamente este
posicionamiento: describe mecanismos funcionales posibles sin prescribir
normativamente que deban ser promovidos.

\textbf{1.2. Problema de investigación: contaminación de varianza y
sesgo de resultado único}

La investigación sobre la Tétrada Oscura ha conceptualizado
tradicionalmente estos rasgos como predictores homogéneamente negativos
asociados con agresión, explotación y daño interpersonal (Paulhus \&
Williams, 2002; Moshagen et al., 2018). No obstante, varios metaanálisis
y trabajos recientes indican que:

\begin{quote}
Existe un núcleo común oscuro o Factor General Antagónico (G),
conceptualizado como la tendencia a maximizar la propia utilidad
aceptando o incluso provocando perjuicio para otros, apoyada en
creencias justificadoras (Moshagen et al., 2018; Zettler et al., 2021).
Evidencia reciente a gran escala sugiere que este núcleo puede
descomponerse en un conjunto restringido de temas funcionales
---insensibilidad, engaño, derecho narcisista, sadismo y venganza--- que
capturan la esencia compartida de los distintos rasgos oscuros (Bader et
al., 2021). Asimismo, se ha mostrado que este componente central compite
con los rasgos específicos por la varianza explicada y suele emerger
como el predictor más robusto de conductas socialmente aversivas, lo que
indica que muchos efectos atribuidos a rasgos particulares podrían ser,
en realidad, manifestaciones de este antagonismo general (Vize et al.,
2020; Hilbig et al., 2021).
\end{quote}

Sobre este núcleo común se superponen componentes específicos de cada
rasgo (e.g., grandiosidad narcisista, cálculo maquiavélico), con
posibles implicancias diferenciales para el desempeño y la innovación
(Miller, 2015; Brownell et al., 2021).

Dos limitaciones metodológicas han impedido clarificar el rol funcional
de esta agencia oscura en contextos organizacionales complejos:

\textbf{Contaminación de varianza:} la mayoría de estudios utiliza sumas
o promedios de escalas de Tétrada/Tríada, sin separar el Factor General
Antagónico (\(G\)) de los componentes agénticos más instrumentales. Dada
la alta colinealidad entre los rasgos, es probable que la varianza
asociada al núcleo malévolo enmascare cualquier contribución positiva de
rasgos más estratégicos (Moshagen et al., 2018; Schreiber \& Marcus,
2020).

Por ello, cuando en esta investigación se habla de ``Agencia Oscura''
residual no se está rebautizando como ``oscura'' una combinación de
rasgos luminosos (e.g., extraversión, consciencia) ni una motivación de
logro convencional. El constructo se define como una forma de
instrumentalidad amoral: una disposición relativamente estable a tratar
normas y personas como medios estratégicos, suspendiendo las
consideraciones morales en la medida en que interfieran con metas de
estatus, control o acceso privilegiado a recursos. A diferencia del
núcleo antagónico general, que implica una orientación explícita al daño
o la explotación, la Agencia Oscura describe la frialdad calculadora que
permite decidir qué reglas se pueden doblar, a quién conviene apoyar y
cuándo conviene retirarse, sin experimentar necesariamente hostilidad
abierta. El modelo Bifactor S-1 se emplea precisamente para aislar
estadísticamente esta capacidad de cálculo del componente de
resentimiento, sadismo o venganza indiscriminada.

\textbf{Sesgo de resultado único:} buena parte de la literatura se
centra en resultados exclusivamente negativos (CWB) o exclusivamente
positivos (desempeño), lo que impide detectar perfiles ``mixtos''. Un
individuo con alta agencia oscura puede innovar internamente y, a la
vez, incurrir en conductas que vulneran reglas burocráticas (CWB-O)
(Pletzer, 2021).

Investigaciones recientes han cuestionado la estructura estándar de la
SD4 y han propuesto soluciones alternativas a los problemas de
colinealidad y solapamiento conceptual. Por un lado, modelos como el
``Dark Five'' fragmentan y reetiquetan los rasgos oscuros, añadiendo
dimensiones adicionales para capturar matices específicos (Crawford et
al., 2025; Dulović et al., 2026). Por otro lado, la propuesta Hateful
Eight (H8) descompone la SD4 en ocho facetas jerárquicamente organizadas
bajo los cuatro rasgos y un factor global, mostrando que el mismo
conjunto de 28 ítems puede puntuarse de forma más eficiente y facetada
(Webster \& Wongsomboon, 2020; Webster et al., 2025). Estos desarrollos
aportan evidencia sólida sobre la complejidad estructural de la Tétrada
Oscura, pero lo hacen principalmente dentro del mismo espacio malévolo
general, ya sea incrementando la dimensionalidad (Dark Five) o
reorganizando facetas (H8), sin separar conceptualmente la ambición
estratégica de la malevolencia básica.

Frente a estas soluciones basadas en la proliferación de factores o
facetas, este estudio aborda los mismos problemas estructurales mediante
una reespecificación explícita basada en un modelo Bifactor S-1 que
separa el Factor General Antagónico (\(G\)) de un componente agéntico
residual (\(S_{Agencia}\)), preservando la parsimonia conceptual y la
utilidad aplicada en contextos organizacionales (Eid et al., 2017;
Bornovalova et al., 2020; Rodriguez et al., 2016). En lugar de adoptar
estos esquemas alternativos, el presente estudio opta por una solución
parsimoniosa centrada en la distinción entre un factor general
antagónico (\(G\)) y un componente agéntico residual (\(S_{Agencia}\))
dentro de la SD4, que resulta más directamente aplicable al análisis de
innovación y desviación en contextos organizacionales.

En este marco, la tesis asume tres niveles de contribución teórica:

\textbf{Contribución principal (núcleo):} clarificar cómo se relacionan
G (núcleo antagónico general) y un componente agéntico residual
(\(S_{Agencia}\)) con EIB, CWB-O y CWB-I en un contexto de baja
institucionalidad y alta politización percibida.

\textbf{Contribución secundaria:} explorar si una forma específica de
Vigilancia Estratégica del Entorno (VEE) funciona como vía disposicional
concurrente que conecta \(S_{Agencia}\) con el EIB.

\textbf{Contribución terciaria, explícitamente exploratoria:}
identificar, mediante Análisis de Perfiles Latentes, patrones de perfil
que sugieran configuraciones cercanas a la ``desviación constructiva
orientada al logro''.

\textbf{1.3. Preguntas de investigación}

Desde este planteamiento, las preguntas de investigación se organizan en
tres ejes:

\textbf{Eje1: validez incremental y diferencial}

\begin{enumerate}
\def\labelenumi{\arabic{enumi}.}
\item
  ¿El factor residual de Agencia Oscura (\(S_{Agencia}\)) ---definido
  como la varianza compartida por narcisismo y maquiavelismo una vez
  controlado el Factor General Antagónico (\(G\))--- muestra validez
  predictiva incremental sobre el EIB y las Conductas Contraproducentes
  Organizacionales (CWB-O), más allá del efecto de \(G\) y de las
  variables de control?
\item
  ¿Cómo se diferencian los patrones de asociación de \(S_{Agencia}\) y
  \(G\) con las dimensiones Organizacional (CWB-O) e Interpersonal
  (CWB-I) de la conducta contraproducente?
\end{enumerate}

\textbf{Eje 2: mecanismo explicativo}

\begin{enumerate}
\def\labelenumi{\arabic{enumi}.}
\setcounter{enumi}{2}
\item
  ¿Se asocia \(S_{Agencia}\) con el EIB a través de una forma específica
  de Vigilancia Estratégica del Entorno (VEE), conceptualmente cercana a
  Strategic Environmental Scanning, entendida como escaneo activo de
  asimetrías de poder e información política (Ashford \& Black, 1996;
  Tang et al., 2012)?
\end{enumerate}

\textbf{Eje 3: condiciones límite}

\begin{enumerate}
\def\labelenumi{\arabic{enumi}.}
\setcounter{enumi}{3}
\item
  ¿Cómo interactúan la Percepción de Política Organizacional (POPS) y el
  Capital Psicológico (PsyCap) para modular la expresión de esta agencia
  oscura, favoreciendo la desviación constructiva orientada al logro
  frente a formas más claramente destructivas de conducta desviada?
\end{enumerate}

\textbf{1.4. Objetivo general y objetivos específicos}

\textbf{Objetivo general}

Evaluar el papel de la Agencia Oscura (\(S_{Agencia}\)) ---residual a un
Factor General Antagónico (\(G\))--- en la predicción conjunta del
comportamiento intraemprendedor (EIB) y de las Conductas
Contraproducentes Organizacionales (CWB-O), considerando la VEE como
mediador disposicional concurrente y la POPS y el PsyCap como
moderadores en trabajadores del sector servicios en Lima, Perú (Gawke et
al., 2019; Bennett \& Robinson, 2000; Hochwarter et al., 2003; Luthans
et al., 2007).

\textbf{Objetivos específicos}

\begin{enumerate}
\def\labelenumi{\arabic{enumi}.}
\item
  Estimar un modelo Bifactor S-1 sobre la Short Dark Tetrad (SD4) que
  permita diferenciar empíricamente el Factor General Antagónico (\(G\))
  y el factor específico de Agencia Oscura (\(S_{Agencia}\)) (Eid et
  al., 2017; Paulhus et al., 2021).
\item
  Analizar la asociación diferencial de \(S_{Agencia}\) y \(G\) con el
  EIB y con las dimensiones Organizacional (CWB-O) e Interpersonal
  (CWB-I) de la conducta contraproducente (Bennett \& Robinson, 2000;
  Pletzer, 2021).
\item
  Examinar si la VEE media, en términos de configuración disposicional
  concurrente, la relación entre \(S_{Agencia}\) y el EIB (Ashford \&
  Black, 1996; Tang et al., 2012).
\item
  Evaluar los efectos moderadores de la POPS y, de forma complementaria,
  del PsyCap sobre los vínculos entre \(S_{Agencia}\), VEE y EIB,
  enmarcando las interacciones de PsyCap como análisis adicionales y no
  como eje central único del modelo (Hochwarter et al., 2003; Luthans et
  al., 2007; Berisha et al., 2025).
\item
  Explorar, mediante Análisis de Perfiles Latentes (LPA), la posible
  existencia de un perfil de ``desviación constructiva orientada al
  logro'', caracterizado por alto \(S_{Agencia}\), \(G\) moderado, EIB
  elevado, CWB-O moderada y CWB-I baja, reconociendo el carácter
  estrictamente exploratorio de este análisis (Nylund et al., 2007;
  Malisetty \& Kumari, 2018).
\end{enumerate}

\textbf{CAPÍTULO 2. MARCO TEÓRICO E HIPÓTESIS}

\textbf{2.1. Vacíos institucionales, informalidad y contextos
politizados}

La literatura sobre emprendimiento en economías emergentes ha mostrado
de manera consistente que los llamados vacíos institucionales no
equivalen a una ausencia de reglas, sino a la coexistencia tensa de
marcos formales débiles con normas informales, arreglos relacionales y
regímenes regulatorios fragmentados que estructuran el comportamiento
económico de forma desigual (Webb et al., 2009, 2013; Welter \&
Smallbone, 2011). En estos contextos, las instituciones formales
---sistemas judiciales, marcos regulatorios, agencias supervisoras---
proporcionan señales ambiguas, intermitentes o selectivas, mientras que
las instituciones informales ---redes, favores, reputación, lealtades---
adquieren un peso desproporcionado en la asignación de oportunidades y
en la gestión de riesgos.

Desde la perspectiva clásica de la asignación del emprendimiento, el
talento emprendedor se distribuye de manera relativamente constante en
la población, pero la dirección productiva, improductiva o abiertamente
destructiva de su uso depende de la estructura de incentivos y
restricciones generada por las instituciones (Baumol, 1990). La
tradición estructuralista latinoamericana ya había anticipado este punto
al subrayar que las economías periféricas se organizan alrededor de una
heterogeneidad estructural persistente, donde segmentos modernos de alta
productividad coexisten con amplios bolsillos de baja productividad e
informalidad, que operan bajo reglas de juego muy diferentes (Pinto,
1970; Pinto \& Di Filippo, 1979). En este marco, la informalidad no es
un mero residuo transitorio, sino un componente funcional del patrón de
acumulación: un amortiguador de costos laborales y un espacio de
resolución ``a la peruana'' de tensiones distributivas (Jiménez, 2010).

El caso peruano ilustra bien esta configuración. Por un lado, la
evidencia macro indica que el país ha experimentado fases de crecimiento
significativo con escasa transformación productiva profunda,
reproduciendo una estructura dual en la que una fracción de firmas y
trabajadores se integra al circuito moderno, mientras amplias capas
permanecen en actividades de baja productividad, alta precariedad y
escasa protección institucional (Jiménez, 2010). Por otro lado, estudios
socioeconómicos sobre el trabajo en el Perú han mostrado que amplios
segmentos de la fuerza laboral operan en una ``economía heterogénea y
marginal'', caracterizada por ingresos inestables, arreglos
contractuales débiles y fuerte dependencia de redes familiares y
comunitarias para gestionar el riesgo (Cavero, 2017). En este entorno,
la frontera entre empleo formal e informal, cumplimiento y evasión,
innovación y pura supervivencia se vuelve borrosa.

En términos institucionales, esto se traduce en un régimen de
cumplimiento selectivo de las normas: las regulaciones existen, pero su
aplicación es desigual, negociada y vulnerable a la captura. Informes
recientes sobre la economía peruana han enfatizado que esta selectividad
no solo es jurídica, sino política: la capacidad de hacer valer o eludir
normas depende de la posición de los actores en redes de poder, de su
acceso a información privilegiada y de su habilidad para operar en zonas
grises (Jiménez, 2010; Francke, 2022). En paralelo, los indicadores
internacionales de calidad institucional ubican al Perú en posiciones
rezagadas en materia de corrupción percibida y estado de derecho,
reforzando la idea de un entorno donde las reglas formales conviven con
amplios espacios de arbitrariedad (Transparency International, 2024;
World Justice Project, 2024).

En años recientes, la noción de vacíos institucionales ha sido matizada
por propuestas que enfatizan la complejidad más que el vacío. Luo (2022)
propone entender muchos entornos emergentes como sistemas de alta
complejidad institucional, en los que coexisten y se solapan múltiples
órdenes normativos ---ley formal, prácticas administrativas, normas
informales, expectativas socioculturales--- que envían señales en parte
contradictorias. En lugar de un ``vacío'' uniforme, los actores
enfrentan un mosaico de reglas parciales, zonas grises y espacios de
ambigüedad donde el cumplimiento es selectivo y negociado.
Psicológicamente, esta complejidad se experimenta como presión cotidiana
para navegar la ambigüedad mediante astucia política, lectura fina del
contexto y construcción activa de coaliciones.

Esta complejidad institucional tiene una traducción directa en la
experiencia psicológica cotidiana de los trabajadores. Las reglas
formales ya no funcionan como un marco único de previsibilidad, sino
como un repertorio parcial que coexiste con ``regímenes de excepción''
informalmente negociados. La Percepción de Política Organizacional
(POPS) deja entonces de ser un simple indicador de clima tóxico para
convertirse en la forma subjetiva que adopta esa arquitectura de poder:
percibir que el avance depende de alianzas, favores y acceso a
información privilegiada equivale a reconocer cuál es la constitución
real de la organización, más allá de sus manuales. En esta tesis, la
POPS se interpreta como el correlato micro de esa complejidad
institucional: un índice de cuán densas y decisivas se perciben las
redes políticas internas para sobrevivir, innovar y capturar
oportunidades en un entorno donde el cumplimiento normativo rara vez es
suficiente.

En el nivel meso-organizacional, esta estructura institucional se
traduce en climas percibidos de alta política y baja previsibilidad.
Cuando los empleados observan que las decisiones clave ---promociones,
asignaciones de recursos, sanciones--- dependen menos de criterios
técnicos y más de vínculos personales, favoritismos o juegos de poder,
tienden a interpretar el entorno como un campo político donde la
neutralidad es ilusoria (Hochwarter et al., 2003; Vigoda-Gadot, 2007).
La Percepción de Política Organizacional (POPS) captura precisamente
esta lectura subjetiva de las ``reglas del juego'' internas: refleja la
sensación de que el avance y la supervivencia dependen de comprender y
gestionar las dinámicas políticas, más que de un desempeño objetivamente
evaluado (Atinc et al., 2010; Kacmar \& Carlson, 1997).

En contextos como el sector de servicios en el Perú ---marcado por
relaciones intensamente relacionales, presencia de intermediación,
presión por resultados a corto plazo y marcos regulatorios fragmentados
(Williams \& Martinez, 2014; INEI, 2024)--- resulta razonable asumir que
la POPS constituye la manifestación psicológica proximal de un entorno
institucional distal caracterizado por informalidad y debilidad
normativa. Es decir, los vacíos y la complejidad institucional no operan
como abstracciones macro, sino como experiencias cotidianas de
arbitrariedad, necesidad de ``moverse con cuidado'' y dependencia de
redes informales.

Desde la \emph{Trait Activation Theory}, estas condiciones contextuales
funcionan como señales situacionales que activan determinados rasgos
disposicionales (Tett \& Burnett, 2003; Judge \& Zapata, 2015). Un
entorno altamente politizado puede operar como contexto ``débil'' en
algunos sentidos ---porque las reglas formales son ambiguas--- pero
también como contexto ``fuerte'' para ciertos rasgos, en la medida en
que envía señales claras sobre qué tipo de conductas son necesarias para
sobrevivir y prosperar. Para individuos con alta orientación a la
justicia y cooperación, un clima de política intensa puede vivirse como
amenaza moral y fuente de cinismo; para perfiles con elevada Agencia
Oscura, el mismo clima puede ser interpretado como una oportunidad
estratégica para explotar asimetrías de información, construir capital
político y avanzar proyectos propios.

En este escenario, la innovación corporativa deja de ser un resultado
casi mecánico de procesos organizacionales estandarizados y pasa a
depender en mayor medida de la agencia individual discrecional. Mantener
comportamientos intraemprendedores (Employee Intrapreneurship Behavior,
EIB) ---explorar, proponer e implementar ideas nuevas dentro de la
organización (Gawke et al., 2019)--- exige navegar simultáneamente
restricciones formales, expectativas informales y juegos de poder. La
informalidad y la politización no solo introducen riesgos para quienes
innovan, sino que también abren espacios de maniobra para quienes están
dispuestos a tensar o reinterpretar reglas, siempre que logren construir
protección política suficiente.

En suma, la combinación de vacíos institucionales, informalidad
persistente y contextos organizacionales percibidos como altamente
politizados configura un entorno donde la transgresión normativa no es
un evento marginal, sino un componente estructural del funcionamiento
cotidiano. Este trabajo asume que, bajo estas condiciones, ciertos
perfiles de personalidad ---particularmente aquellos asociados a la
Agencia Oscura--- pueden encontrar un nicho funcional: su disposición a
detectar y explotar asimetrías de poder, a operar en zonas grises y a
sostener estrategias de riesgo elevado puede traducirse tanto en
innovación intraemprendedora como en conductas contraproducentes.
Aclarar cómo se articula esta agencia oscura con la estructura
institucional descrita constituye el punto de partida para el desarrollo
de las secciones siguientes sobre personalidad oscura, vigilancia
estratégica del entorno y desviación constructiva en organizaciones de
servicios peruanas.

\textbf{2.2. Personalidad oscura, Factor} \(D\) \textbf{y Agencia
Oscura}

El concepto de Tríada Oscura ---narcisismo, maquiavelismo y
psicopatía--- fue introducido para describir un estilo interpersonal
frío y manipulador, asociado con autopromoción, baja empatía y conductas
socialmente aversivas (Paulhus \& Williams, 2002). Investigaciones
posteriores ampliaron este conjunto hacia una Tétrada Oscura,
incorporando el sadismo como rasgo adicional (Buckels et al., 2013).

Metaanálisis recientes indican que estos rasgos comparten un núcleo
común ---el Dark Factor of Personality (\(D\))--- definido como la
tendencia a maximizar la propia utilidad aceptando o provocando daño
hacia los demás, apoyada en creencias justificadoras (Moshagen et al.,
2018; Zettler et al., 2021). Este núcleo se asocia consistentemente con
agresión, explotación, baja honestidad-humildad y otros resultados
disfuncionales en el trabajo y la sociedad (Moshagen et al., 2018;
Jonason \& Webster, 2010). Clarificaciones meta-analíticas recientes
muestran que la varianza compartida entre los rasgos oscuros es tan
sustancial que, en muchos contextos, el Factor \(D\) explica mejor las
conductas socialmente aversivas que los rasgos por separado. Schreiber y
Marcus (2020) advierten que los modelos que ignoran este núcleo común
corren el riesgo de atribuir indebidamente al maquiavelismo, al
narcisismo o a la psicopatía efectos que, en realidad, son expresión de
una comorbilidad antagónica general.

En este proyecto, dicho núcleo se conceptualiza de manera amplia como
Dark Factor of Personality (\(D\)) y se operacionaliza, específicamente
dentro de la SD4, como un Factor General Antagónico (\(G\)) que captura
esta varianza común de antagonismo. Es decir, G funciona como
representación latente del componente oscuro general en el espacio
medido por la SD4, mientras que los factores específicos o residuales
permiten examinar qué queda de los rasgos oscuros una vez aislado este
núcleo compartido (Moshagen et al., 2018; Zettler et al., 2021).

En estudios longitudinales recientes se ha demostrado que este núcleo
oscuro posee validez predictiva incremental significativa sobre formas
de psicopatología socialmente aversiva, más allá de los rasgos
individuales específicos (Hilbig et al., 2021). Esto refuerza la premisa
central de este proyecto: sin aislar la varianza de este núcleo general,
cualquier intento de estimar la contribución relativamente ``funcional''
o agéntica de la Tétrada Oscura corre el riesgo de estar contaminado por
la toxicidad antagónica basal.

Sin embargo, más allá de este núcleo general, subsisten facetas
específicas con posibles implicancias funcionales diferenciadas. Por
ejemplo, el narcisismo grandioso se ha vinculado con carisma, ambición y
liderazgo en contextos de alta visibilidad, aunque también con
sobreconfianza y riesgo excesivo (Chatterjee \& Hambrick, 2007; Miller,
2015). El maquiavelismo se caracteriza por cálculo frío, orientación al
largo plazo y habilidad para manipular información y relaciones, rasgos
que pueden ser adaptativos en entornos políticamente complejos,
dependiendo de sus objetivos y restricciones (Jones \& Paulhus, 2009).

En el plano de la estructura interna, estudios recientes han mostrado
que la Tétrada Oscura presenta una organización claramente jerárquica y
multifacética. La propuesta Hateful Eight (H8; Webster \& Wongsomboon,
2020, 2025) ilustra este enfoque al derivar ocho facetas a partir de la
SD4 ---por ejemplo, \emph{deviousness} y \emph{scheming} para el
maquiavelismo, \emph{leadership} y \emph{exceptionalism} para el
narcisismo, \emph{defiance} y \emph{recklessness} para la psicopatía, y
\emph{violent voyeurism} y \emph{verbal abuse} para el sadismo--- que se
agrupan bajo los cuatro rasgos oscuros y, a su vez, bajo un factor
global. La versión actualizada del H8, basada en múltiples estudios,
refuerza la idea de que los puntajes de la SD4 capturan un entramado
jerárquico de facetas específicas, rasgos de orden superior y un núcleo
oscuro general. Este tipo de evidencia respalda que los rasgos oscuros
no son bloques unitarios, sino configuraciones complejas donde coexisten
componentes más abiertamente antagónicos con otros de carácter agéntico
o instrumental.

En el contexto peruano, la Short Dark Tetrad (SD4) ya cuenta con
evidencia psicométrica específica. Zegarra-López y colegas (2024)
adaptaron la SD4 a una muestra de adultos peruanos, mostrando un ajuste
adecuado de modelos jerárquicos e índices de fiabilidad adecuados para
los cuatro rasgos, así como indicadores de invarianza por sexo y edad.
Los autores concluyen que la estructura de la Tétrada Oscura en Perú
reproduce un núcleo oscuro común con componentes diferenciados, lo que
respalda conceptualmente la decisión de este estudio de modelar un
Factor General Antagónico (\(G\)) junto con un componente agéntico
específico \(S_{Agencia}\). De forma complementaria, estudios recientes
de redes con muestras peruanas (Ramos-Vera et al., 2023, 2024) han
mostrado que los ítems vinculados con dureza emocional, manipulación
estratégica y derecho narcisista tienden a ocupar posiciones centrales y
de puente dentro del sistema de rasgos oscuros, lo que sugiere que estos
contenidos constituyen focos dinámicos de integración del ``lado
oscuro'' de la personalidad en contextos latinoamericanos. Esta
evidencia local refuerza la pertinencia de diferenciar empíricamente
entre un núcleo antagónico general y un componente de agencia oscura
residual al analizar innovaciones y transgresiones normativas en
organizaciones peruanas.

De forma convergente, un análisis de redes reciente sobre la Tríada
Oscura y la inteligencia emocional en adultos peruanos muestra que los
contenidos vinculados con dureza emocional, manipulación estratégica y
falta de empatía ocupan posiciones centrales y de puente en el sistema
de rasgos, conectando los nodos oscuros con déficits en competencias
socioemocionales (Ramos-Vera et al., 2023). Este patrón sugiere que, en
el contexto peruano, la integración dinámica del ``lado oscuro'' no se
reduce a una simple falta de amabilidad, sino, más bien, que se fragua
una arquitectura de frialdad instrumental que puede resultar funcional
para navegar entornos politizados, pero al precio potencial de
deteriorar la calidad relacional y la contención emocional en el
trabajo. Esta evidencia respalda la decisión de este estudio de aislar
un componente de Agencia Oscura residual a G, precisamente porque dicho
componente parece concentrar los contenidos estratégicos y fríos que se
sitúan en el ``nudo'' de la red de rasgos.

Ahora bien, los hallazgos de redes en muestras peruanas sugieren que
esta arquitectura no se organiza solo alrededor de la dureza emocional
en abstracto, sino de contenidos muy concretos vinculados con gestionar
información sensible y con postergar retaliaciones. Los ítems que aluden
a guardar datos para utilizarlos en el momento oportuno o a esperar
pacientemente la ocasión adecuada para vengarse tienden a ocupar
posiciones privilegiadas en la red, lo que sugiere que el eje
estructurante del ``lado oscuro'' local es una lógica de control del
tiempo y del acceso a información. Bajo esta lectura, la Agencia Oscura
residual deja de confundirse con una ambición ``neutral'' y se entiende
como la capacidad de leer y explotar un sistema donde callar, filtrar
selectivamente y dosificar castigos simbólicos son recursos centrales de
influencia. Esta forma de instrumentalidad amoral es la que, en
contextos de vacíos institucionales, podría conectarse con innovaciones
intraemprendedoras de alto riesgo, pero también con patrones específicos
de transgresión burocrática.

Modelos teóricos recientes ---incluido el denominado modelo Gray Nine,
que propone una taxonomía de nueve rasgos ``grises'' intermedios entre
lo adaptativo y lo claramente maladaptativo--- cuestionan la distinción
binaria entre rasgos ``adaptativos'' y ``maladaptativos'' y plantean que
determinados rasgos oscuros pueden mostrar patrones funcionales en
configuraciones específicas de contexto y de rasgos (Boman, 2024). En
línea con esta perspectiva, el presente proyecto se centra en un
componente de Agencia Oscura depurado de \(D\)/\(G\) y evalúa
empíricamente si dicho componente residual puede asociarse con
resultados adaptativos en entornos organizacionales de baja
institucionalidad, especialmente en términos de innovación
intraemprendedora, aun cuando su perfil siga siendo ambivalente en
términos normativos.

El proyecto, por tanto, se inserta en la línea general que reconoce la
naturaleza multifacética y jerárquica de la Tétrada Oscura, pero se
distancia de las estrategias puramente facetarias (e.g., incrementar el
número de factores) al focalizarse específicamente en aislar, dentro de
dicho espacio oscuro, un componente de Agencia Oscura residual al núcleo
antagónico general, con potencial funcional para la innovación
intraorganizacional.

En este sentido, la tesis se sitúa deliberadamente dentro del giro
reciente de la psicología de la personalidad que aboga por modelos
integrados y jerárquicos, más que por taxonomías rivales y fragmentadas.
Las revisiones panorámicas del campo han subrayado que la proliferación
de etiquetas (Tríada, Tétrada, Dark Five, \(D\), etc.) ha generado más
ruido que claridad si no se explicita qué nivel de abstracción se está
modelando y con qué fin aplicado (Rauthmann, 2021; Roberts et al.,
2024). Lejos de proponer una ``nueva marca'' de rasgos, este proyecto
adopta una postura conservadora: asume la existencia de un núcleo oscuro
amplio (\(D\)/\(G\)) bien documentado, reconoce la evidencia de
estructuras jerárquicas complejas y se concentra en una pregunta mucho
más modesta y acotada: qué queda de la agencia oscura cuando el
antagonismo general se controla explícitamente en un contexto
organizacional concreto.

Trabajos recientes ilustran la necesidad de este tipo de precisiones.
Rose y colegas han mostrado que, en modelos comparativos, buena parte de
la supuesta ``novedad'' de algunos constructos oscuros se evapora cuando
se les enfrenta a medidas bien especificadas de antagonismo y
honestidad-humildad: al descomponer la varianza, muchas escalas terminan
midiendo variaciones sobre un mismo eje central de dureza interpersonal
(Rose et al., 2021). De forma complementaria, Horsten et al. han
enfatizado que la utilidad de los modelos oscuros depende menos de
agregar etiquetas (Tríada vs. Tétrada vs. \(D\)) que de aclarar qué
parte de la varianza compartida y específica se pone a trabajar en
predicciones concretas de conducta (Horsten et al., 2023). Esta tesis
recoge esa advertencia: no pretende competir con el Factor \(D\), sino
operar ``dentro de él'', separando el antagonismo general de un
componente agéntico residual y preguntando en qué condiciones
contextuales este último puede asociarse simultáneamente a innovación
intraemprendedora y transgresión burocrática.

En esa línea, el modelo defendido aquí se apoya explícitamente en la
cartografía conceptual y metodológica trazada por estas contribuciones
de alto nivel. Las discusiones sobre el ``\emph{Landscape}'' de la
psicología de la personalidad han insistido en la necesidad de vincular
los debates estructurales con preguntas sustantivas sobre qué hacen los
rasgos en contextos sociales específicos (Beck \& Jackson, 2022). El
presente trabajo asume esa agenda: combina la claridad estructural de
los modelos de núcleo oscuro (Moshagen et al., 2018; Zettler et al.,
2021) con la sensibilidad contextual de la investigación sobre
emprendimiento en vacíos institucionales y maverickismo organizacional
(Webb et al., 2013; Jordan et al., 2021, 2023, 2025). En lugar de
reivindicar originalidad nominal, la tesis declara sus deudas: se apoya
en estos ``hombros de gigantes'' para responder una pregunta aplicada
precisa sobre innovación, desviación y agencia oscura en organizaciones
de servicios peruanas.

\textbf{2.3. Reespecificación bifactorial del SD4:} \(G\) \textbf{y}
\(S_{Agencia}\)

Para operacionalizar la distinción entre antagonismo general y agencia
oscura, se adopta un modelo Bifactor S-1 (Eid et al., 2017). En este
enfoque:

\begin{itemize}
\item
  Un \emph{Factor General Antagónico} \textbf{(}\(G\)\textbf{)},
  definido por los ítems de psicopatía y sadismo como dominios de
  referencia, captura los contenidos más claramente malévolos y
  explotadores (baja honestidad-humildad, crueldad, instrumentalización
  del otro).
\item
  Un \emph{Factor Específico de Agencia Oscura}
  \textbf{(}\(S_{Agencia}\)\textbf{)}, por su parte, se define como la
  varianza residual compartida por los ítems de narcisismo y
  maquiavelismo una vez controlada la varianza atribuible a G. Este
  factor representa la combinación de antagonismo con componentes
  instrumentales y estratégicos (búsqueda de estatus, cálculo frío,
  manipulación relacional).
\end{itemize}

Dado que la estimación de factores específicos en modelos bifactoriales
puede ser exigente, se establece un protocolo de contingencia: si el
modelo Bifactor S-1 no converge o muestra problemas de identificación,
se utilizará un enfoque de residualización observada (regresión del
índice de Agencia ---Narcisismo + Maquiavelismo--- sobre el índice de
Antagonismo). De este modo, se preserva la lógica conceptual (aislar un
componente agéntico residual) aunque la estimación latente plena resulte
inviable.

Esta distinción conceptual se implementará empíricamente mediante la
Short Dark Tetrad (SD4; Paulhus et al., 2021), especificada bajo un
modelo Bifactor S-1 cuyos detalles se presentan en el Capítulo 3. Una
parte de la contribución empírica de la tesis consiste precisamente en
comprobar si la estructura se aproxima a una organización casi
unidimensional (dominio de \(G\)) o si \(S_{Agencia}\) retiene varianza
común interpretable y útil.

\textbf{2.4. Comportamiento intraemprendedor (EIB) y conducta
contraproducente (CWB)}

El comportamiento intraemprendedor (EIB) se conceptualiza como un
conjunto de conductas individuales de exploración, generación e
implementación de nuevas ideas de negocio, productos, servicios o
procesos dentro de la organización (Gawke et al., 2019). A nivel
empírico, se ha demostrado que el EIB se relaciona con desempeño,
innovación organizacional y ventaja competitiva, especialmente en
sectores intensivos en conocimiento y servicios (Gawke et al., 2019).

En esta línea, resulta útil concebir el comportamiento intraemprendedor
como una forma específica de desviación constructiva. La literatura
sobre emprendimiento ha subrayado que iniciar o impulsar cambios
relevantes implica, casi inevitablemente, romper hábitos, cuestionar
normas y traspasar límites establecidos, incluso cuando la intención
subyacente sea funcional para la organización. Malisetty y Kumari (2018)
proponen precisamente entender el emprendimiento como un tipo particular
de conducta desviada, donde la frontera entre ``desviación deseable'' e
``indeseable'' no puede definirse solo en función de las intenciones,
dado que los resultados pueden ser inciertos o ambivalentes. Desde esta
perspectiva, el EIB se aproxima a una desviación constructiva orientada
al logro, mientras que las CWB ---especialmente cuando expresan
antagonismo interpersonal--- encarnan formas más claramente destructivas
de transgresión.

En paralelo, la conducta contraproducente en el trabajo (CWB) comprende
acciones voluntarias que vulneran normas organizacionales y amenazan el
bienestar de la organización o de sus miembros (Robinson \& Bennett,
1995). Aunque se ha tendido a conceptualizarla como una consecuencia
relativamente monolítica de la personalidad oscura, metaanálisis
recientes sugieren una relación más matizada. Pletzer (2021) muestra que
no todos los componentes oscuros se asocian de igual manera con las
distintas formas de CWB: mientras el antagonismo general predice de
forma robusta conductas dañinas interpersonales e impulsivas, los
componentes más agénticos pueden vincularse con transgresiones
normativas de carácter instrumental.

En este proyecto, esta distinción resulta central: la transgresión
burocrática organizacional (CWB-O) puede ser utilizada estratégicamente
por perfiles con elevada agencia oscura para ``atajar'' la burocracia y
sortear obstáculos, diferenciándose cualitativamente de la desviación
maliciosa orientada a dañar directamente a colegas (CWB-I). La
distinción entre CWB-O y CWB-I es, por tanto, clave: la hipótesis de
transgresión funcional supone que ciertos individuos podrían vulnerar
principalmente normas burocráticas (CWB-O) en aras de implementar
cambios o sortear obstáculos, sin necesariamente mostrar niveles
equivalentes de daño interpersonal (baja CWB-I).

Esta ambivalencia entre innovación y daño normativo encuentra eco en
evidencia latinoamericana reciente. Un estudio con 550 trabajadores de
Perú y Ecuador mostró que el maquiavelismo y la psicopatía se asocian
negativamente con las conductas socialmente responsables en el lugar de
trabajo y con la preocupación ambiental, de modo que los perfiles más
oscuros tienden a mostrar menor involucramiento en prácticas sostenibles
y orientadas al bien común (Bortoluzzi et al., 2023). Estos resultados
indican que, incluso cuando ciertos componentes de la personalidad
oscura puedan facilitar la transgresión instrumental de reglas
burocráticas para ``hacer que las cosas sucedan'', existe un riesgo real
de que esa misma energía se traduzca en desatención de compromisos
sociales y ambientales clave.

La tesis se inserta precisamente en esta tensión: explorar cuándo la
Agencia Oscura se canaliza hacia desviación constructiva orientada al
logro y cuándo deriva en deterioro de la responsabilidad organizacional
en un subcontexto donde la muestra incluía una proporción relevante de
trabajadores peruanos del sector servicios.

\textbf{2.4.1. Maverickismo, contrarianismo y desviación positiva}

La literatura clásica sobre personalidad oscura y conducta desviada ha
tendido a tratar la no conformidad como un fenómeno relativamente
homogéneo: quienes desafían normas son, en principio, problemáticos para
la organización. Sin embargo, trabajos recientes sobre no conformidad y
estereotipos sociales muestran que bajo la etiqueta de ``desviados''
conviven perfiles psicológicos y funcionales marcadamente distintos
(Haasl et al., 2025). Esta distinción es clave para un modelo que
pretende diferenciar daño organizacional, daño interpersonal y
desviación constructiva asociada a formas instrumentales de agencia
oscura.

Haasl y colegas (2025) distinguen entre dos tipos de no conformistas:
los \emph{mavericks}, impulsados principalmente por la búsqueda de
independencia, y los \emph{contrarians}, impulsados por el deseo de ser
distintos al resto. En una serie de estudios con población
estadounidense, los mavericks son estereotipados como altamente
competentes y concienzudos, adecuados para roles de liderazgo
transformacional, más probablemente varones, de mayor edad y
relativamente satisfechos con su vida. En cambio, los contrarians son
vistos como muy sociables pero bajos en calidez y amabilidad, más
neuróticos y orientados hacia roles que enfatizan la creatividad y la
autoexpresión, siendo percibidos como más jóvenes y menos satisfechos
con su vida (Haasl et al., 2025). Esta taxonomía social sugiere que no
toda desviación respecto a las normas comparte la misma cualidad moral
ni las mismas implicaciones funcionales.

Más allá de los estereotipos, la literatura sobre maverickismo lo
conceptualiza como un patrón de diferencias individuales caracterizado
por competencia social, creatividad, foco en metas, propensión al riesgo
y conductas disruptivas (Gardiner \& Jackson, 2015). En un estudio con
trabajadores a tiempo completo, Gardiner y Jackson (2015) muestran que
el maverickismo se asocia a alta extraversión, alta apertura y baja
agradabilidad, y que ciertos procesos de aprendizaje autorregulado
explican varianza adicional en maverickismo por encima de los Cinco
Grandes. Es decir, los mavericks no solo difieren en rasgos
descriptivos, sino también en sus estilos de exploración e improvisación
creativa, combinando un potencial claro para la innovación radical con
una probabilidad no trivial de generar fricción con normas y
autoridades.

Este doble filo del maverickismo se ha reinterpretado desde la noción de
desviación positiva. Jordan y colegas argumentan que los mavericks
ejemplifican una forma de ``non-conformidad acotada'' (\emph{bounded
non-conformity}): violan normas locales de procedimiento, jerarquía o
estilo, pero permanecen alineados con hipernormas organizacionales de
orden superior, como el desempeño colectivo o la integridad del sistema
(Jordan et al., 2021). Desde esta óptica, parte de su conducta desviada
constituye desviación positiva, en la medida en que rompe reglas para
proteger valores que la propia organización declara centrales.

Trabajos posteriores muestran que estos actores acumulan combinaciones
específicas de capital cultural, social y simbólico que les permiten
sostener su desviación en contextos fuertemente normativos (Jordan et
al., 2023, 2025). Este ``capital maverick'' aumenta la probabilidad de
que su disrupción sea tolerada e incluso valorada cuando se alinea con
narrativas de cambio estratégico o mejora de desempeño.

Integrando estas líneas, el presente proyecto asume que ciertos perfiles
de personalidad oscura pueden aproximarse funcionalmente al
maverickismo, mientras que otros se asemejan más al contrarianismo en el
sentido de Haasl et al. (2025). De forma puramente conceptual ---ya que
la tesis no mide directamente maverickismo ni contrarianismo---, puede
denominarse ``maverickismo oscuro'' a la combinación de: (a)
disposiciones instrumentales de la personalidad oscura (por ejemplo,
frialdad estratégica, disposición a explotar asimetrías de información),
(b) rasgos maverick relacionados con creatividad, orientación a metas y
tolerancia al riesgo (Gardiner \& Jackson, 2015), y (c) capital político
y comprensión de las reglas formales e informales de la organización, en
línea con los recursos descritos por Jordan et al. (2023, 2025). Bajo
condiciones de vacíos institucionales, alta percepción de política
organizacional y presiones de cambio, este perfil podría canalizarse en
formas de desviación positiva que combinan comportamiento
intraemprendedor y violaciones instrumentales de normas orientadas al
logro de resultados.

En contraste, un ``contrarianismo oscuro'' combinaría motivaciones
centradas en la autodiferenciación y la oposición visible al
\emph{status quo} con rasgos oscuros orientados al dominio relacional y
la búsqueda de estatus, coherente con la caracterización de los
contrarians como altamente sociables pero bajos en calidez y
agradabilidad, y más neuróticos (Haasl et al., 2025). En contextos de
fuerte ambigüedad política, este perfil sería más proclive a
manifestarse como conducta contraproducente interpersonal (CWB-I),
conflicto, cinismo y sabotaje simbólico que como desviación funcional
orientada al desempeño organizacional.

La literatura sobre el ``bright side'' de la personalidad oscura
converge con esta distinción funcional. Estudios multiculturales sobre
la Tríada Oscura y el desempeño laboral han mostrado que, una vez
controlados los grandes rasgos de personalidad, ciertos componentes
narcisistas y maquiavélicos pueden relacionarse positivamente ---aunque
con tamaños de efecto modestos--- con indicadores de desempeño y
contribuciones extra-rol, especialmente en roles que requieren
asertividad, negociación dura y tolerancia al conflicto (Ma et al.,
2021). En el ámbito emprendedor latino, la evidencia sobre rasgos
oscuros y emprendimiento naciente sugiere que estos perfiles pueden
inclinar a los individuos hacia estrategias oportunistas de captura de
nichos en entornos institucionalmente frágiles (Afshar Jahanshahi et
al., 2025). El concepto de ``maverickismo oscuro'' recoge precisamente,
el punto de la presente investigación: el problema no es tanto la
desviación en sí misma, sino el nivel de alineamiento entre la
transgresión y las hipernormas organizacionales (desempeño,
sostenibilidad, integridad).

En suma, la distinción maverick/contrarian, junto con la literatura
sobre maverickismo y desviación positiva (Gardiner \& Jackson, 2015;
Jordan et al., 2021, 2023, 2025), sirve en esta tesis como lente
conceptual para refinar la noción de ``agencia oscura''. La pregunta
empírica central es si el componente \(S_{Agencia}\) ---residual a
\(G\)--- se comporta más como una forma de no conformidad orientada al
logro (maverickismo oscuro) o como una oposición principalmente
destructiva (contrarianismo oscuro) cuando se relaciona con EIB, CWB-O y
CWB-I en contextos de baja institucionalidad.

\textbf{2.5. Vigilancia estratégica del entorno (VEE)}

La literatura sobre proactividad laboral ha descrito conductas de
búsqueda de \emph{feedback}, información y oportunidades como
componentes clave de la adaptación organizacional (Ashford \& Black,
1996). No obstante, muchas escalas clásicas capturan una forma
relativamente ``benigna'' de búsqueda de información (por ejemplo,
clarificar expectativas, solicitar apoyo).

En el contexto de este estudio, la VEE se concibe como un patrón de
escaneo orientado a mapear la distribución de poder y recursos,
identificar vulnerabilidades estructurales y localizar oportunidades que
dependen de saber quién controla qué recursos. Este estilo de vigilancia
se alinea con la lógica de la agencia oscura: no se trata de curiosidad
epistémica general, sino de una búsqueda instrumental de asimetrías de
información que puedan ser explotadas para obtener ventaja competitiva o
avanzar proyectos intraemprendedores bajo condiciones de incertidumbre
(Rauch \& Frese, 2007; Baker \& Nelson, 2005).

En términos operativos, la VEE se aproxima tanto a formas de vigilancia
política intraorganizacional como a patrones más generales de alerta
emprendedora ante oportunidades y riesgos, siempre que impliquen un
componente de escaneo activo del entorno (Tang et al., 2012). Para
preservar la validez discriminante respecto al comportamiento
intraemprendedor (EIB), la VEE se definirá y operará empíricamente como
un constructo fundamentalmente orientado al escaneo de poder, recursos y
vulnerabilidades estructurales, más que a la generación directa de ideas
o a la mejora de procesos.

En este trabajo, la VEE se concibe menos como un monitoreo pasivo del
entorno y más como un proceso de mapeo cognitivo de vacíos y asimetrías.
No se limita a ``mirar alrededor'', sino a cartografiar qué normas se
aplican de forma estricta, cuáles son letra muerta, quiénes actúan como
nodos de protección o sanción y qué combinaciones de actores permiten
experimentar con soluciones no convencionales. Para los perfiles con
alta Agencia Oscura, este escaneo se articula con una flexibilidad moral
selectiva: la información recopilada sirve para resignificar ciertas
transgresiones como arbitraje normativo racional más que como simple
desviación. De este modo, la VEE reduce la ansiedad de aprendizaje que
suelen generar contextos hostiles y abre espacio para la exploración
conductual: permite probar atajos burocráticos, reinterpretar
procedimientos o acelerar decisiones, calculando fríamente los riesgos
políticos y las probabilidades de impunidad.

\textbf{2.6. Percepción de política organizacional (POPS) y \emph{Trait
Activation}}

La POPS refiere a la percepción de que las decisiones organizacionales
se basan en intereses personales, favoritismos y juegos de poder más que
en criterios objetivos (Hochwarter et al., 2003). La literatura suele
asociar altos niveles de POPS con cinismo, estrés y disminución del
compromiso, especialmente en perfiles orientados a la justicia y la
cooperación (Vigoda-Gadot, 2007).

Desde la \emph{Trait Activation Theory}, sin embargo, estas mismas
señales pueden activar rasgos que se expresan precisamente en contextos
de ambigüedad y competencia política (Tett \& Burnett, 2003; Judge \&
Zapata, 2015). Para individuos con elevada agencia oscura, un entorno
percibido como politizado puede funcionar como ``señal de oportunidad'',
incrementando la vigilancia estratégica y la búsqueda de información
relevante sobre redes, alianzas y amenazas.

\textbf{2.7. Capital psicológico (PsyCap) y Conservación de Recursos}

El Capital Psicológico (PsyCap) integra cuatro recursos psicológicos
positivos: autoeficacia, esperanza, optimismo y resiliencia (Luthans et
al., 2007). Metaanálisis recientes muestran que el PsyCap se asocia de
forma consistente con actitudes laborales positivas, desempeño y
conductas de ciudadanía organizacional (Avey et al., 2011; Luthans et
al., 2010).

En contextos de alta incertidumbre y presión estructural, el PsyCap
adquiere una función especialmente crítica. Estudios sobre
emprendimiento y sostenibilidad han mostrado que la autoeficacia, la
esperanza y la resiliencia facilitan que los individuos mantengan el
esfuerzo emprendedor frente a restricciones institucionales,
\emph{shocks} externos y crisis prolongadas (Tang, 2020; Milosevic et
al., 2017). En el plano intraorganizacional, evidencias recientes
sugieren que el PsyCap de directivos y mandos intermedios se relaciona
de manera robusta con la intención intraemprendedora y con la
disposición a explorar nuevas oportunidades, tanto de forma directa como
a través del bienestar psicológico (Berisha et al., 2025).

En contextos de alta politización y vacíos institucionales, este recurso
``luminoso'' adquiere una dimensión ambivalente. Combinado con niveles
elevados de Agencia Oscura, el PsyCap no solo amortigua el desgaste
emocional asociado a la incertidumbre, sino que provee el colchón
subjetivo necesario para sostener estrategias de transgresión
burocrática de largo aliento. Un trabajador con mucha esperanza y
resiliencia, pero también con fuerte orientación instrumental al
estatus, puede perseverar en la búsqueda de atajos normativos allí donde
otros desistirían, reinterpretando su audacia como ingenio más que como
conducta éticamente problemática. Desde la lógica de la Teoría de
Conservación de Recursos, ello implica que el PsyCap facilita tomar
riesgos normativos porque incrementa la percepción de contar con
recursos suficientes para absorber pérdidas o sanciones potenciales. En
esta tesis se asume explícitamente esta doble cara: el PsyCap se modela
como un habilitador potencial tanto de desviación constructiva orientada
al logro como, en ausencia de contrapesos, de trayectorias más
corrosivas.

Simultáneamente, la literatura sobre comportamiento desviado indica que
la inseguridad laboral y la angustia psicológica incrementan la
probabilidad de conductas contraproducentes, especialmente cuando las
reglas se perciben como injustas o arbitrarias (Dang-Van et al., 2022).
En este escenario, el PsyCap no solo funciona como un ``stock'' de
recursos, sino como un mecanismo regulador que puede canalizar la
tendencia a romper reglas hacia la innovación intraemprendedora,
amortiguando el salto hacia formas más claramente destructivas de
desviación.

Desde la \emph{Conservation of Resources Theory} (COR), los individuos
tienden a acumular, proteger y expandir recursos; cuando disponen de un
``stock'' elevado, es más probable que se involucren en conductas
arriesgadas o demandantes, ya que perciben capacidad para afrontar
posibles pérdidas (Hobfoll, 2001). Aplicado a este estudio, el PsyCap
podría funcionar como recurso habilitante que permite a la agencia
oscura sostener comportamientos intraemprendedores sin colapsar ante el
riesgo o la resistencia interna.

De forma coherente con esta evidencia, el presente estudio asume un rol
compensatorio del PsyCap en contextos politizados y de baja justicia.
Cuando las estructuras formales proveen poco soporte, los recursos
psicológicos positivos se vuelven más determinantes para decidir si la
energía disruptiva asociada a la Agencia Oscura se traduce en innovación
o en conducta contraproducente (Gojny-Zbierowska, 2024).

Además de la adaptación al español del PCQ-12 disponible en la
literatura, el PsyCap cuenta ya con evidencia específica en el contexto
peruano. Guerrero-Alcedo y Espina-Romero (2024), utilizando un enfoque
bayesiano con estudiantes universitarios peruanos, mostraron que una
versión abreviada del cuestionario de capital psicológico presenta buen
ajuste de modelo, alta fiabilidad interna y validez convergente con
indicadores de bienestar, así como ausencia de problemas graves de
funcionamiento diferencial de los ítems. Estos resultados sugieren que
el PsyCap es un recurso psicológico pertinente y psicométricamente
sólido para poblaciones peruanas, incluso bajo condiciones de alta
incertidumbre estructural. En consecuencia, el presente estudio se apoya
tanto en las adaptaciones previas al español del PCQ-12 como en esta
evidencia local reciente para justificar el uso del PsyCap como
moderador y recurso habilitante clave en contextos de baja
institucionalidad y alta percepción de política organizacional.

\textbf{2.8. Hipótesis}

A partir del marco teórico, se formulan las siguientes hipótesis,
distinguiendo entre contribuciones centrales y análisis complementarios
de carácter exploratorio. Dado el diseño transversal, todas las
hipótesis se interpretan en términos de asociaciones estructurales
concurrentes, no de efectos causales temporales estrictos.

\textbf{H1. Patrones diferenciales de resultados (paradoja de la
agencia) -- Núcleo del modelo}

\begin{itemize}
\item
  \textbf{H1a.} Controlando por el Factor General Antagónico (\(G\)), se
  espera que el componente residual de Agencia Oscura (\(S_{Agencia}\))
  presente una asociación positiva con el comportamiento
  intraemprendedor (EIB).
\item
  \textbf{H1b.} Controlando por \(G\), se espera que \(S_{Agencia}\)
  muestre una asociación positiva con las Conductas Contraproducentes
  Organizacionales (CWB-O), especialmente aquellas que implican
  transgresión burocrática instrumental, mientras que su asociación con
  las Conductas Contraproducentes Interpersonales (CWB-I) será nula o
  significativamente menor en magnitud.
\item
  \textbf{H1c.} Se espera que \(G\) se asocie negativamente con el EIB y
  positivamente con ambas dimensiones de CWB (organizacional e
  interpersonal), reflejando un patrón más homogéneamente destructivo.
\end{itemize}

\textbf{H2. Vigilancia estratégica como mediación disposicional
concurrente -- Contribución secundaria}

\begin{itemize}
\item
  \textbf{H2.} Se espera una asociación indirecta positiva de
  \(S_{Agencia}\) sobre el EIB a través de la Vigilancia Estratégica del
  Entorno (VEE), interpretada como una configuración disposicional
  concurrente: los individuos con mayor Agencia Oscura tenderían
  simultáneamente a reportar niveles superiores de VEE y de
  comportamiento intraemprendedor.
\end{itemize}

\textbf{H3. POPS como señal de activación (\emph{Trait Activation
Theory}) -- Moderador principal}

\begin{itemize}
\item
  \textbf{H3.} Se espera que la Percepción de Política Organizacional
  (POPS) modere positivamente la relación entre \(S_{Agencia}\) y la
  VEE, de modo que la asociación \(S_{Agencia} \rightarrow \ VEE\) sea
  más fuerte cuando el entorno se percibe como altamente politizado. En
  términos de Trait Activation, un clima político más intenso
  funcionaría como señal situacional que facilita la expresión de la
  agencia oscura en forma de vigilancia estratégica.
\end{itemize}

\textbf{H4. PsyCap como recurso habilitante (\emph{Conservation of
Resources Theory}) -- Hipótesis complementaria}

\begin{itemize}
\item
  \textbf{H4 (complementaria).} Se explorará si el Capital Psicológico
  (PsyCap) modera positivamente la relación entre \(S_{Agencia}\) y el
  comportamiento intraemprendedor (EIB). Se espera que esta función
  movilizadora del PsyCap sea más crítica precisamente cuando el entorno
  se percibe como altamente politizado o injusto, actuando como recurso
  compensatorio que favorece que la energía disruptiva asociada a la
  Agencia Oscura se canalice hacia la innovación en lugar de hacia la
  desviación destructiva (Dang-Van et al., 2022; Gojny-Zbierowska, 2024;
  Tang, 2020; Berisha et al., 2025). Dado el estado aún incipiente de la
  evidencia, esta hipótesis se formula como análisis complementario.
\end{itemize}

\textbf{H5. Heterogeneidad latente (Análisis de Perfiles Latentes) --
Análisis exploratorio}

\begin{itemize}
\item
  \textbf{H5 (exploratoria).} Se explorará, mediante Análisis de
  Perfiles Latentes (LPA), la posible existencia de un perfil de
  ``desviación constructiva orientada al logro'', caracterizado por alto
  \(S_{Agencia}\), \(G\) moderado, EIB alto, CWB-O moderado y CWB-I
  baja. La identificación, tamaño y estabilidad de este perfil se
  considerarán hallazgos exploratorios y no constituirán criterio de
  confirmación del modelo teórico principal.
\end{itemize}

En síntesis, el modelo teórico propuesto puede representarse como una
arquitectura de doble capa. En la primera, estructural, el Factor
General Antagónico (\(G\)) y el componente residual de Agencia Oscura
(\(S_{Agencia}\)) predicen de forma diferenciada tres resultados clave:
comportamiento intraemprendedor (EIB), conducta contraproducente
organizacional (CWB-O) y conducta contraproducente interpersonal
(CWB-I).

En la segunda capa, disposicional--contextual, la Vigilancia Estratégica
del Entorno (VEE) funciona como mecanismo mediador concurrente entre
\(S_{Agencia}\) y el EIB, cuya activación depende de señales
situacionales de política percibida (POPS) y de la disponibilidad de
recursos psicológicos positivos (PsyCap).

Sobre este entramado continuo se superpone, a nivel exploratorio, una
capa de heterogeneidad latente que distingue perfiles combinados de
\(G\), \(S_{Agencia}\), EIB, CWB-O y CWB-I ---incluido un potencial
perfil de ``desviación constructiva orientada al logro''---, lo que
permite evaluar si la paradoja de la agencia oscura se manifiesta en
configuraciones de rasgos y comportamientos coherentes en contextos de
baja institucionalidad.

\textbf{2.9. Formulación matemática del modelo estructural}

Para explicitar la lógica del modelo propuesto en términos formales, se
representa la estructura básica mediante un sistema de ecuaciones
estructurales.

Sea \(G\) el Factor General Antagónico, \(S_{A}\) el factor específico
de Agencia Oscura (residual a \(G\) en el modelo bifactorial sobre la
SD4), \(VEE\) la Vigilancia Estratégica del Entorno, \(EIB\)el
comportamiento intraemprendedor, \(CWB_{O}\) la conducta
contraproducente organizacional, \(CWB_{I}\) la conducta
contraproducente interpersonal, \(POPS\) la percepción de política
organizacional y \(PsyCap\) el capital psicológico.

Las relaciones estructurales centrales pueden resumirse como sigue:

\[{VEE = \alpha_{0} + \alpha_{1}S_{A} + \alpha_{2}POPS + \alpha_{3}(S_{A} \times POPS) + \varepsilon_{VEE}
}{EIB = \beta_{0} + \beta_{1}S_{A} + \beta_{2}G + \beta_{3}VEE + \beta_{4}PsyCap + \beta_{5}(S_{A} \times PsyCap) + \varepsilon_{EIB}
}{CWB_{O} = \gamma_{0} + \gamma_{1}S_{A} + \gamma_{2}G + \varepsilon_{CWB_{O}}
}{CWB_{I} = \delta_{0} + \delta_{1}S_{A} + \delta_{2}G + \varepsilon_{CWB_{I}}
}\]donde los términos producto \(S_{A} \times POPS\) y
\(S_{A} \times PsyCap\) representan las interacciones moderadoras
postuladas en las hipótesis H3 y H4, respectivamente.

En el caso del modelo latente (Escenario A), estas ecuaciones se
estimarán con variables latentes y términos de interacción latente
mediante métodos de máxima verosimilitud robusta con integración
numérica (por ejemplo, enfoque LMS), mientras que en el modelo observado
(Escenario B) se implementarán mediante términos producto centrados
sobre índices compuestos (Marsh et al., 2004; Muthén \& Asparouhov,
2006).

La mediación disposicional concurrente postulada en H2 se formaliza
mediante el producto \(\alpha_{1} \times \beta_{3}\), que representa el
efecto indirecto de \(S_{A}\)sobre \(EIB\)a través de \(VEE\). Este
efecto se estimará mediante bootstrapping no paramétrico con 5 000
remuestreos, obteniendo intervalos de confianza sesgo--corregidos.
Finalmente, el componente exploratorio de heterogeneidad latente (H5) se
implementará mediante Análisis de Perfiles Latentes (LPA) sobre las
variables (\(G,S_{A},EIB,CWB_{O},CWB_{I}\)), lo que permite verificar si
la combinación de parámetros estimados en el modelo continuo se refleja
también en configuraciones cualitativamente diferenciables de individuos
(Nylund et al., 2007).

En suma, el modelo propone que la Agencia Oscura (\(S_{Agencia}\)),
residual al Factor General Antagónico (\(G\)), se relaciona
simultáneamente con innovación intraemprendedora (EIB) y transgresión
burocrática (CWB-O), mientras que \(G\) se asocia de forma más
homogéneamente destructiva con la conducta contraproducente
interpersonal (CWB-I). Además, se plantea que la Vigilancia Estratégica
del Entorno (VEE) constituye el mecanismo disposicional que conecta
\(S_{Agencia}\) con el EIB, y que tanto la percepción de política
organizacional (POPS) como el capital psicológico (PsyCap) modulan la
expresión de esta agencia oscura, favoreciendo, en ciertas condiciones,
formas de desviación constructiva orientada al logro.

Una implicación central de este modelo es que la Vigilancia Estratégica,
por sí sola, no garantiza la innovación. La diferencia entre un empleado
prudente que observa el entorno para protegerse y un Agente Oscuro que
lo explota estratégicamente reside en una proclividad específica a
actuar sobre las oportunidades que emergen de las zonas grises. El
presente estudio asume que la VEE, cuando se combina con altos niveles
de Agencia Oscura y con recursos psicológicos suficientes, habilita una
forma de ``desviación constructiva orientada al logro'': conductas que
tensan o reconfiguran reglas burocráticas internas con el fin de crear
valor para la organización y, simultáneamente, capturar beneficios de
estatus, acceso a recursos o protección política para el propio agente.
Esta proclividad a intervenir activamente sobre el vacío institucional
constituye el puente teórico entre la personalidad oscura y la
innovación intraemprendedora en contextos de baja institucionalidad.

\textbf{CAPÍTULO 3. METODOLOGÍA}

\textbf{3.1. Diseño}

El estudio adopta un diseño cuantitativo, no experimental, transversal y
correlacional-explicativo. Se utilizarán modelos de ecuaciones
estructurales (SEM) para contrastar las relaciones propuestas entre
variables latentes y, en caso necesario, análisis de rutas (\emph{path
analysis}) con variables observadas residuales. Dado el uso de
autoinformes para todas las variables sustantivas, se considerará
explícitamente el riesgo de varianza del método común (CMV) y se
incorporarán tanto estrategias procedimentales como estadísticas para
mitigar su impacto (Podsakoff et al., 2003; Spector, 2006). Estos
procedimientos se entenderán como medios para evaluar y acotar la
influencia potencial de la CMV, más que como garantía de su eliminación
completa (Podsakoff et al., 2003; Spector, 2006).

Los modelos se estimarán mediante máxima verosimilitud robusta (por
ejemplo, estimador MLR), considerando la no normalidad potencial de los
datos, y el tratamiento de datos perdidos se realizará mediante máxima
verosimilitud con información completa (FIML), asumiendo un mecanismo de
ausencia al menos \emph{missing at random} (Enders, 2010; Kline, 2016).
En todos los modelos se emplearán estimadores robustos a desviaciones de
normalidad (por ejemplo, MLR en Mplus), apropiados para ítems tipo
Likert y estructuras complejas tratando los ítems tipo Likert como
variables continuas, en línea con trabajos de simulación que muestran un
desempeño aceptable de MLR con escalas de 5 puntos y tamaños muestrales
grandes.

En consonancia con la naturaleza transversal del diseño, los modelos de
mediación se interpretarán estrictamente en términos de asociaciones
disposicionales concurrentes, más que como cadenas causales temporales.
Es decir, se examinará si los individuos con mayor Agencia Oscura
tienden simultáneamente a reportar niveles superiores de vigilancia
estratégica del entorno (VEE) y de comportamiento intraemprendedor
(EIB), siguiendo recomendaciones recientes sobre el uso de modelos de
mediación en diseños no longitudinales (Spector, 2019).

\textbf{3.2. Participantes y muestreo}

La población objetivo está compuesta por trabajadores del sector
servicios (finanzas, telecomunicaciones, consultoría, seguros y
servicios empresariales) en Lima Metropolitana. Este sector concentra
una proporción creciente del empleo urbano y se caracteriza por una alta
intensidad relacional y márgenes significativos de discrecionalidad en
la ejecución del rol (Instituto Nacional de Estadística e Informática
{[}INEI{]}, 2024).

El contexto institucional peruano presenta tres rasgos estructurales que
sustentan la noción de ``vacíos institucionales''. En primer lugar, la
tasa de informalidad laboral se mantiene elevada en el sector servicios
(INEI, 2024). En segundo lugar, los indicadores de calidad institucional
sitúan al Perú en posiciones rezagadas, con altos niveles de corrupción
percibida en el sector público (Transparency International, 2024). En
tercer lugar, el Índice de Estado de Derecho reporta puntuaciones bajas
en cumplimiento regulatorio (World Justice Project, 2024). Sobre esta
base, se argumenta que la Percepción de Política Organizacional (POPS)
constituye la manifestación psicológica proximal de un entorno
institucional distal caracterizado por informalidad y debilidad
normativa.

Adicionalmente, estudios recientes realizados con trabajadores de Perú y
otros países latinoamericanos muestran que los rasgos de la Tríada
Oscura ya están siendo empleados como predictores de comportamientos
laborales críticos, como la responsabilidad social en el trabajo y las
intenciones emprendedoras (Afshar Jahanshahi et al., 2023; Afshar
Jahanshahi et al., 2025). Esto sugiere que el contexto de Lima
Metropolitana no constituye un caso exótico, sino un ejemplo
paradigmático de economías de servicios en entornos de baja
institucionalidad donde la personalidad oscura se entrelaza con
estrategias de adaptación laboral y emprendimiento.

Se utilizará un muestreo no probabilístico por conveniencia, reclutando
participantes a través de redes de contacto (estrategias de bola de
nieve controlada) y organizaciones del sector servicios. En
consecuencia, las inferencias se circunscriben a trabajadores de
características similares, y no se reclama representatividad estadística
de todo el sector. Se proyecta un tamaño muestral mínimo de
\(N \geq 500\) trabajadores, justificado mediante análisis a priori de
potencia para detectar efectos pequeños en interacciones
(\(f² \approx .02,\ \alpha = .05,\ 1\  - \ \beta = .80\)) en modelos
estructurales con múltiples parámetros (Wolf et al., 2013). Aquel tamaño
se considera suficiente para estimar modelos estructurales de
complejidad moderada y explorar interacciones y perfiles latentes. No
obstante, los análisis más complejos (e.g., LPA) se interpretarán con
especial cautela, atendiendo a tamaños de clase y estabilidad de
soluciones (Nylund et al., 2007).

\textbf{3.3. Instrumentos}

\textbf{3.3.1. Rasgos oscuros de la personalidad: Agencia Oscura y
Antagonismo (SD4)}

Para la medición de los rasgos oscuros se empleará la versión adaptada
al contexto peruano de la Short Dark Tetrad (SD4) (Paulhus et al., 2021;
Zegarra-López et al., 2024). Este instrumento consta de 28 ítems
distribuidos equitativamente en cuatro subescalas: maquiavelismo,
narcisismo, psicopatía y sadismo. El estudio de adaptación peruana ha
mostrado propiedades psicométricas sólidas, con evidencias de buena
consistencia interna e invarianza por grupos etarios en adultos
peruanos, lo que respalda su uso en contextos organizacionales locales
(Zegarra-López et al., 2024). En paralelo, versiones hispanohablantes
recientes de la SD4 han confirmado la estructura jerárquica de la
Tétrada Oscura y la presencia de un núcleo oscuro común en muestras
adultas de habla hispana (Ortet-Walker et al., 2024).

Sin embargo, en este proyecto los puntajes de la SD4 no se tratarán como
simples sumas por subescala. Con el fin de aislar empíricamente un
componente de Agencia Oscura distinto del Factor General Antagónico
(\(G\)), se especificará un modelo de medida Bifactor S-1 (Eid et al.,
2017). En este enfoque, los ítems de psicopatía y sadismo se utilizarán
como dominio de referencia del factor general G, que capturará los
contenidos más claramente malévolos y explotadores (baja
honestidad-humildad, crueldad, instrumentalización del otro); todos los
ítems de la SD4 cargarán en este factor general. Adicionalmente, los
ítems de narcisismo y maquiavelismo cargarán en un factor específico de
Agencia Oscura (\(S_{Agencia}\)), definido como la varianza residual
compartida entre estos ítems una vez controlada la varianza atribuible a
G.

La idoneidad del modelo Bifactor S-1 se evaluará mediante índices
globales de ajuste (CFI, TLI, RMSEA, SRMR) y, específicamente, mediante
indicadores diseñados para estructuras bifactoriales: Varianza Común
Explicada (ECV), omega jerárquico (\(\omega H\)) del factor general y
Porcentaje de Correlaciones no Contaminadas (PUC) (Rodríguez et al.,
2016; Bornovalova et al., 2020). Estos indicadores permitirán determinar
si la SD4 se comporta como una estructura esencialmente unidimensional
dominada por \(G\) o si el factor específico \(S_{Agencia}\) retiene
suficiente varianza y fiabilidad residual como para ser interpretado de
forma sustantiva.

La elección del modelo Bifactor S-1 no responde únicamente a criterios
de ajuste estadístico, sino al problema conceptual planteado en el
Capítulo 1: evitar que el núcleo antagónico general ``se coma'' la
posible funcionalidad de los componentes agénticos. Al fijar uno de los
factores ---en este caso, el Factor General Antagónico--- como
referencia global y modelar la Agencia Oscura como factor específico
residual, se restringe la varianza compartida atribuible a la
malevolencia básica y se somete a prueba la hipótesis fuerte de que
persiste un componente instrumental diferenciable. De este modo, el
contraste entre G y la Agencia Oscura no se apoya solo en argumentos
teóricos, sino en una partición explícita de la varianza: si el factor
residual no logra predecir nada una vez controlado G, la tesis debe
aceptar la crítica de que la Agencia Oscura es un ``fantasma
estadístico''.

Más allá de los índices globales de ajuste y de los indicadores
específicos para modelos bifactoriales (ECV, \(\omega H\), PUC), la
adecuación del componente específico de Agencia Oscura (\(S_{Agencia}\))
se evaluará también en términos de la descomposición de la varianza de
la matriz de covarianzas observada (\(\Sigma\)). En particular, se
examinará la proporción de varianza común atribuible al Factor General
Antagónico (\(G\)) frente a la varianza residual retenida por
\(S_{Agencia}\), así como el patrón de autovalores asociado al espacio
residual una vez parcializada la contribución de \(G\). Este análisis
permitirá comprobar que \(S_{Agencia}\) no constituye un factor residual
numéricamente identificable pero empíricamente trivial, sino que captura
una fracción sustantiva y estable de la varianza compartida por los
ítems de narcisismo y maquiavelismo. En línea con los criterios clásicos
para la evaluación de estructuras factoriales en análisis multivariado
(Mardia, Kent, \& Bibby, 1979; Johnson \& Wichern, 2007) y con las
recomendaciones recientes sobre el uso responsable de modelos
bifactoriales en psicopatología y personalidad (Rodríguez et al., 2016;
Bornovalova et al., 2020), se evitará deliberadamente la sobreextracción
de factores específicos con autovalores marginales o interpretabilidad
dudosa. Solo en la medida en que \(S_{Agencia}\) muestre cargas
factoriales sustantivas, fiabilidad residual aceptable y contribución
apreciable a la varianza común, se justificará su inclusión como factor
diferenciado en el modelo estructural.

Se contemplarán dos escenarios analíticos:

\begin{itemize}
\item
  Escenario A (modelo bifactor latente operativo).
\end{itemize}

\begin{quote}
El modelo Bifactor S-1 convergerá adecuadamente y mostrará buen ajuste
global, con cargas factoriales sustantivas y fiabilidad al menos
moderada para \(S_{Agencia}\) (\(\omega \approx .60 - .70\), aceptable
dada la brevedad de la escala y el carácter residual del factor). En
este escenario, G y S\_Agencia se incorporarán como factores latentes
diferenciados en el modelo estructural principal.
\end{quote}

\begin{itemize}
\item
  Escenario B (estructura esencialmente unidimensional).
\end{itemize}

\begin{quote}
El modelo Bifactor S-1 presentará problemas de convergencia, cargas
débiles para \(S_{Agencia}\) o evidencias de un dominio prácticamente
unidimensional. En este caso, se construirá un índice observado de
Agencia (suma de ítems de narcisismo y maquiavelismo) y un índice
observado de Antagonismo (psicopatía + sadismo u otra combinación
respaldada por la literatura de la Tétrada Oscura), y el índice de
Agencia se regresionará sobre el índice de Antagonismo. El residuo
estandarizado resultante (\(S_{Res}\)) se interpretará como aproximación
observada de la Agencia Oscura residual a \(G\) y se utilizará como
predictor en un modelo de rutas con variables observadas.
\end{quote}

Este enfoque permite alinear la operacionalización de la SD4 con la
evidencia reciente sobre el Factor \(D\) y el núcleo oscuro común de la
personalidad (Moshagen et al., 2018; Zettler et al., 2021), a la vez que
preserva la posibilidad de identificar un componente agéntico residual
diferenciado, coherente con la noción de Agencia Oscura planteada en el
marco teórico.

\textbf{3.3.2. Vigilancia estratégica del entorno (VEE)}

La VEE se operacionalizará como un patrón de escaneo intraorganizacional
orientado a identificar asimetrías de poder, vacíos normativos y
vulnerabilidades estructurales relevantes para la acción. La escala se
construirá combinando ítems de búsqueda de información organizacional
derivados del cuestionario de proactividad durante la entrada
organizacional de Ashford y Black (1996) ---específicamente la
subdimensión de búsqueda de información estructural y política--- con
ítems de la dimensión de exploración y búsqueda de la escala de alerta
emprendedora de Tang, Kacmar y Busenitz (2012), adaptados al contexto
intraorganizacional. En el caso de Ashford y Black (1996), se tomarán
como referencia los ítems que capturan el esfuerzo por aprender la
estructura formal e informal de la organización, las políticas y los
procedimientos clave, mientras que de Tang et al. (2012) se aprovecharán
ítems que reflejan la búsqueda activa y sistemática de información
relevante para detectar oportunidades y cambios en el entorno.

Todos los ítems se adaptarán al castellano siguiendo un procedimiento de
traducción--retrotraducción con jueces expertos, atendiendo a las
recomendaciones psicométricas actuales para la adaptación de
instrumentos (Muñiz, Elosua, \& Hambleton, 2013). Además, se realizará
una fase piloto con análisis factorial exploratorio para verificar que
los ítems conforman un factor unidimensional interpretable de vigilancia
estratégica, diferenciado de forma clara tanto del comportamiento
intraemprendedor como de la alerta emprendedora genérica. De este modo,
la VEE se definirá empíricamente como un constructo centrado en el
escaneo político--estratégico de poder, recursos y reglas flexibles, más
que en la generación directa de ideas o la mejora de procesos (Ashford
\& Black, 1996; Rauch \& Frese, 2007; Tang et al., 2012).

\textbf{Fase 0: Validación de contenido y análisis factorial
exploratorio.} En esta fase se instruirá a jueces expertos para eliminar
o reescribir cualquier ítem que implique innovación directa (por
ejemplo, ``busco nuevas formas de hacer las cosas''), a fin de proteger
la validez discriminante frente al EIB. Solo se retendrán ítems de
escaneo político/estratégico o de mapeo de poder y recursos.
Posteriormente, se realizará un análisis factorial exploratorio o
confirmatorio, según el tamaño muestral en la fase piloto, para
verificar unidimensionalidad y consistencia interna.

La VEE se evaluará mediante una escala compuesta que integra ítems
procedentes de dos fuentes complementarias. En primer lugar, se
utilizarán los cuatro ítems de búsqueda de información sobre la
estructura y las reglas de la organización de la escala de proactividad
de Ashford y Black (1996) --en su versión traducida al español--,
centrados en el aprendizaje de la estructura formal, las políticas y
procedimientos relevantes y las redes informales de relaciones de poder.
En segundo lugar, se incorporarán ítems seleccionados de la dimensión de
exploración y búsqueda de la escala de alerta emprendedora de Tang et
al. (2012), utilizando la versión en español validada recientemente en
una muestra universitaria peruana en el marco de un proyecto dirigido
por la Universidad Complutense de Madrid. Esta versión castellana aporta
evidencias de validez factorial y fiabilidad en población peruana, lo
que la convierte en una base sólida para el presente estudio. Los ítems
serán adaptados léxicamente para enfatizar el escaneo
intraorganizacional de asimetrías de poder, vacíos normativos y
oportunidades vinculadas a la posición de los actores, manteniendo el
formato de respuesta tipo Likert de 5 puntos (1 = ``completamente en
desacuerdo'' a 5 = ``completamente de acuerdo'') y siguiendo un
procedimiento de traducción--retrotraducción y juicio de expertos para
asegurar la claridad y la equivalencia conceptual respecto a las
versiones originales (Ashford \& Black, 1996; Tang et al., 2012).

Los cuatro ítems de `búsqueda de información' de Ashford y Black (1996)
en su adaptación española (e.g., `Intentado aprender la estructura
oficial de la organización') se seleccionarán y reformularán para
enfatizar el mapeo de poder y reglas informales, mientras que de la
escala de alerta emprendedora de Tang et al. (2012) se utilizarán los
ítems de exploración y búsqueda más centrados en adquisición activa de
información relevante para oportunidades.

\textbf{3.3.3. Comportamiento intraemprendedor (EIB)}

El comportamiento intraemprendedor (EIB) se medirá mediante la
\emph{Employee Intrapreneurship Behavior Scale} (EIB) desarrollada por
Gawke et al. (2019). Esta escala evalúa conductas de exploración,
generación e implementación de nuevas ideas de negocio, productos,
servicios o procesos dentro de la organización, y ha mostrado relaciones
robustas con desempeño, innovación organizacional y ventaja competitiva,
especialmente en sectores intensivos en conocimiento y servicios (Gawke
et al., 2017, 2019).

Dado que la versión original se desarrolló en inglés y en contextos
europeos, se llevará a cabo un proceso de traducción y adaptación
cultural al castellano siguiendo las recomendaciones estándar para la
adaptación de instrumentos psicométricos: traducción directa por
traductores bilingües con formación en psicología organizacional,
síntesis y revisión por un panel experto, retrotraducción independiente
y comparación con la versión original para asegurar equivalencia
semántica y conceptual. Posteriormente, se verificará la estructura
factorial propuesta por Gawke et al. (2019) en la muestra peruana
mediante análisis factorial confirmatorio, evaluando tanto el ajuste
global como la fiabilidad compuesta de las dimensiones.

Aunque la evidencia sobre la EIB en contextos hispanohablantes es
todavía emergente, estudios recientes sobre intraemprendimiento docente
e innovaciones impulsadas desde la base han mostrado que configuraciones
conductuales análogas ---orientadas a la detección de oportunidades y a
la implementación de cambios dentro de organizaciones intensamente
normadas--- pueden modelarse y medirse de forma válida en entornos de
lengua española (p.ej., Arteaga-Reyes et al., 2022). Esta literatura
refuerza la plausibilidad de emplear la EIB como indicador de
comportamiento intraemprendedor en el contexto organizacional peruano.

En coherencia con el marco teórico de esta tesis, en los análisis se
otorgará especial atención a la distinción entre EIB como desviación
constructiva orientada al logro y las formas de conducta
contraproducente (CWB), de modo que los resultados empíricos permitan
evaluar si la Agencia Oscura se asocia a una transgresión esencialmente
funcional o destructiva.

\textbf{3.3.4. Conducta contraproducente en el trabajo (CWB)}

La conducta laboral contraproducente (CWB) se evaluará mediante una
adaptación de la Workplace Deviance Scale de Bennett y Robinson (2000),
que distingue entre conducta contraproducente organizacional (CWB-O) y
conducta contraproducente interpersonal (CWB-I) (Robinson \& Bennett,
1995). Esta escala ha sido ampliamente utilizada en la literatura
internacional para capturar conductas voluntarias que vulneran normas
organizacionales y amenazan el bienestar de la organización o de sus
miembros.

En el contexto peruano, la Workplace Deviance Scale ha sido empleada y
adaptada en estudios recientes con trabajadores de Lima Metropolitana,
en particular en investigaciones sobre el impacto de la supervisión
abusiva y el liderazgo pseudo-transformacional en el comportamiento
laboral contraproducente (Ramírez Vilca et al., 2022).

Dichos trabajos reportan evidencias de consistencia interna adecuada y
respaldan la viabilidad de utilizar esta medida en contextos laborales
peruanos de servicios.

En este estudio, se llevará a cabo un proceso de ajuste y depuración
teóricamente guiado de los ítems. En línea con la noción de ``desviación
constructiva orientada al logro'', los análisis se centrarán en aquellos
ítems de CWB-O que reflejen transgresión activa de normas (por ejemplo,
atajos burocráticos, omisión deliberada de procedimientos considerados
innecesarios) y se tratarán de forma separada ---o se excluirán de los
modelos estructurales principales--- los ítems que reflejen meramente
retirada pasiva del esfuerzo (por ejemplo, llegar tarde, dedicar poco
esfuerzo), con el fin de no confundir Agencia Oscura con baja
consciencia o descompromiso general (Dahling et al., 2012).

Adicionalmente, se realizarán análisis de sensibilidad excluyendo los
ítems de CWB-O de mayor severidad (por ejemplo, daño deliberado a la
propiedad, sabotaje explícito) para comprobar si los patrones de
asociación con \(S_{Agencia}\) se mantienen cuando el foco se limita a
desviaciones de proceso de menor intensidad, pero potencialmente
funcionales. La dimensión CWB-I se conservará completa, dado que
representa el polo claramente destructivo de la transgresión
interpersonales en el modelo (hostilidad, agresión, daño a colegas).

\textbf{3.3.5. Percepción de política organizacional (POPS)}

La Percepción de Política Organizacional (POPS) se medirá mediante la
escala desarrollada por Hochwarter et al. (2003), que captura la
percepción de que las decisiones organizacionales se basan en intereses
personales, favoritismos y juegos de poder más que en criterios
meritocráticos u objetivos. Esta medida se alinea directamente con la
conceptualización de climas altamente politizados utilizada en el marco
teórico y ha mostrado relaciones robustas con cinismo, estrés y
resultados laborales en diversas muestras internacionales (Hochwarter et
al., 2003; Vigoda-Gadot, 2007).

Dado que no existe aún una validación exhaustiva de POPS en el contexto
peruano, se seguirá un procedimiento de traducción y adaptación cultural
análogo al descrito para la EIB: traducción directa, síntesis por panel
experto, retrotraducción y ajuste fino para preservar el significado de
los ítems asociados a favoritismo, juegos de poder y uso instrumental de
las reglas. En una primera etapa, se realizará un análisis factorial
confirmatorio para verificar la estructura de la escala y se evaluará su
consistencia interna en la muestra de trabajadores del sector servicios
de Lima Metropolitana.

La decisión de trabajar exclusivamente con POPS ---sin incorporar
simultáneamente medidas de clima ético como constructo adicional---
obedece a una estrategia de parsimonia teórica y estadística: el modelo
propuesto se centra en el papel de los climas politizados como señal de
activación de la Agencia Oscura (en la lógica de la \emph{Trait
Activation Theory}), por lo que el uso de POPS como indicador central de
las ``reglas del juego'' subjetivamente percibidas resulta coherente con
el foco de la tesis (Tett \& Burnett, 2003; Judge \& Zapata, 2015).

\textbf{3.3.6. Capital psicológico (PsyCap)}

El Capital Psicológico (PsyCap) se evaluará mediante la versión
abreviada de 12 ítems del Psychological Capital Questionnaire (PCQ-12)
(Avey et al., 2011), que integra las dimensiones de autoeficacia,
esperanza, resiliencia y optimismo en un recurso psicológico compuesto
(Luthans et al., 2007). Metaanálisis previos han mostrado asociaciones
consistentes del PsyCap con actitudes laborales positivas, desempeño y
conductas de ciudadanía organizacional en diversos contextos
organizacionales (Avey et al., 2011; Luthans et al., 2010).

En el ámbito hispanohablante, el PCQ-12 ha sido adaptado y validado en
España, mostrando buena fiabilidad y una estructura factorial de segundo
orden estable (León-Pérez et al., 2017). Más recientemente, en el
contexto peruano se ha confirmado la validez de constructo y la
fiabilidad del PCQ-12 en población universitaria, replicando la
estructura de cuatro factores de primer orden y un factor de segundo
orden, así como coeficientes \(\alpha\) y \(\omega\) satisfactorios
(Guerrero, 2024).

En esta tesis se utilizará la versión en castellano del PCQ-12
respaldada por Guerrero-Alcedo y Espina-Romero (2024), ajustando
únicamente la redacción de algunos ítems para adecuarlos al contexto de
trabajadores del sector servicios, sin alterar su contenido semántico
central. La elección de este instrumento se justifica por su brevedad,
evidencia psicométrica local y coherencia conceptual con el rol que el
PsyCap asume en el modelo: un recurso habilitante que puede modular la
forma en que la Agencia Oscura se traduce en comportamiento
intraemprendedor o en conducta contraproducente, especialmente en
entornos percibidos como politizados y de baja justicia (Dang-Van et
al., 2022; Gojny-Zbierowska, 2024).

\textbf{3.3.7. Control de varianza del método común (CMV)}

La variable marcadora (\emph{marker variable}) se diseñará para
compartir el mismo formato de respuesta y nivel de abstracción que las
demás escalas, pero abordando contenidos no relacionados con el contexto
laboral (por ejemplo, actitudes generales hacia temas sociales
relativamente neutros), con el fin de capturar la fracción de varianza
atribuible al método sin introducir solapamiento conceptual sustantivo
(Williams et al., 2010). La variable marcadora se utilizará siguiendo el
enfoque de covarianza parcial propuesto por Williams et al. (2010),
comparando modelos con y sin ajuste por CMV.

Adicionalmente, se considerará el uso de constructos como Afectividad
Negativa y/o Deseabilidad Social como covariables de control en análisis
de sensibilidad, dado que, aunque son buenos candidatos para capturar
estilos de respuesta, también comparten varianza sustantiva con rasgos
oscuros y su parcialización excesiva podría eliminar efectos legítimos
(Podsakoff et al., 2003; Spector, 2006). Por esta razón, dichos
constructos se reservarán exclusivamente para análisis de sensibilidad y
no se incluirán en el modelo principal.

No obstante, se reconoce que en estudios basados en autoinforme la CMV
nunca puede eliminarse por completo. En este proyecto, la defensa
principal frente a este riesgo no descansa solo en la variable
marcadora, sino en la estructura misma de los modelos a estimar. Los
efectos de interacción y las configuraciones de perfiles latentes que se
ponen a prueba (por ejemplo, la combinación específica de alta Agencia
Oscura, VEE elevada, CWB-I baja y CWB-O moderada) son patrones difíciles
de explicar como artefactos triviales de método. Si los resultados
muestran que las asociaciones cambian sistemáticamente según niveles de
POPS y PsyCap, y que emergen clases latentes conceptualmente coherentes,
ello constituye evidencia indirecta de que el modelo está capturando
configuraciones psicológicas significativas más allá de una mera
tendencia de respuesta generalizada o de deseabilidad social uniforme.

\textbf{3.4. Procedimiento}

Los datos se recogerán mediante un cuestionario online anónimo,
difundido en organizaciones del sector servicios en Lima Metropolitana.
Se garantizará el consentimiento informado y la confidencialidad
absoluta de las respuestas, conforme a los estándares éticos de
investigación con seres humanos. Los participantes serán informados de
que su participación es voluntaria, que pueden retirarse en cualquier
momento sin consecuencias y que los datos se analizarán únicamente en
forma agregada.

Cuando sea posible, se recogerá información mínima sobre la organización
(por ejemplo, tamaño, sector específico) sin identificadores directos,
con el fin de explorar, en análisis descriptivos, posibles patrones
diferenciales entre subsectores, sin pretender inferencias multinivel
formales.

\textbf{3.5. Estrategia de análisis de datos}

Previamente a la estimación de los modelos de medida se llevará a cabo
una fase sistemática de depuración y preprocesamiento de los datos,
orientada a identificar tanto patrones de respuesta inatenta (*careless
responding*) como observaciones multivariadamente atípicas. En primer
lugar, se examinarán indicadores descriptivos de calidad de respuesta,
tales como tiempos de respuesta extremadamente reducidos, patrones de
respuesta invariante a lo largo de los ítems (por ejemplo, uso reiterado
de la misma categoría de respuesta) e índices intraindividuo simples
(e.g., longitud de cadenas de respuestas idénticas y varianza
intraindividual). Estos indicadores permitirán detectar casos en los que
la validez del protocolo sea dudosa por falta de implicación mínima del
participante.

En segundo lugar, se evaluará la presencia de valores atípicos
multivariados mediante el cálculo de la distancia de Mahalanobis sobre
el conjunto de puntuaciones de los ítems o subescalas principales (SD4,
EIB, CWB-O, CWB-I, POPS, PsyCap). Los casos con distancias extremas se
identificarán a partir de los valores críticos de la distribución
\(\chi ²\), ajustando el umbral de significación al tamaño muestral y al
número de variables incluidas. La decisión de excluir observaciones se
basará en un criterio combinado estadístico y sustantivo: solo se
eliminarán aquellos protocolos que presenten simultáneamente indicios
claros de respuesta inatenta y valores atípicos multivariados extremos,
con el fin de evitar tanto la retención de datos espurios como la
eliminación indiscriminada de casos legítimos (Johnson \& Wichern,
2007). De este modo, se busca minimizar la distorsión de la estructura
factorial y de las asociaciones estructurales derivada de patrones de
respuesta aberrantes, lo cual resulta especialmente relevante en
contextos de vacíos institucionales y medición de rasgos socialmente
sensibles.

El análisis se organizará en tres fases secuenciales, siguiendo un
enfoque parsimonioso alineado con los objetivos centrales del estudio.

\textbf{Fase 1. Modelo de medida y evaluación de la estructura oscura}

En la primera fase se evaluará el modelo de medición de todas las
variables latentes, con énfasis en la estructura de la SD4. Para los
rasgos oscuros se especificará un modelo Bifactor S-1 (Eid et al.,
2017), donde los ítems de psicopatía y sadismo definen el dominio de
referencia del Factor General Antagónico (\(G\)), y todos los ítems
cargan en dicho factor general; adicionalmente, los ítems de narcisismo
y maquiavelismo cargarán en un factor específico de Agencia Oscura
(\(S_{Agencia}\)), representando la varianza residual compartida una vez
controlado \(G\).

El ajuste se juzgará mediante índices globales convencionales (CFI y TLI
≥ .90, RMSEA y SRMR ≤ .08) y, específicamente para el modelo
bifactorial, mediante indicadores de robustez estructural tales como la
Varianza Común Explicada (ECV), el coeficiente omega jerárquico
(\(\omega H\)) del factor general y el Porcentaje de Correlaciones no
Contaminadas (PUC) (Rodríguez et al., 2016; Bornovalova et al., 2020).
Estos índices permitirán determinar si la varianza común está
fuertemente dominada por \(G\) (estructura esencialmente unidimensional)
o si el factor específico \(S_{Agencia}\) retiene suficiente varianza y
fiabilidad como para ser interpretado.

Se contemplan dos escenarios:

\begin{itemize}
\item
  \textbf{Escenario A (modelo bifactor latente operativo).}
\end{itemize}

\begin{quote}
El modelo Bifactor S-1 converge adecuadamente, presenta buen ajuste
global y el factor específico \(S_{Agencia}\) exhibe cargas factoriales
sustantivas y fiabilidad al menos moderada
(\(\omega\  \approx \ .60 - .70\), aceptable dada la brevedad de la
escala y el carácter residual del factor). En este escenario, \(G\) y
\(S_{Agencia}\) se utilizarán como factores latentes diferenciados en la
Fase 2.
\end{quote}

\begin{itemize}
\item
  \textbf{Escenario B (estructura esencialmente unidimensional).}
\end{itemize}

\begin{quote}
El modelo Bifactor S-1 presenta problemas de convergencia, cargas
débiles para \(S_{Agencia}\) o evidencia de estructura casi
unidimensional. En este caso se construirá un índice observado de
Agencia (Narcisismo + Maquiavelismo) y un índice observado de
Antagonismo (Psicopatía + Sadismo u otra combinación respaldada por la
literatura). El índice de Agencia se regresionará sobre el índice de
Antagonismo, y el residuo estandarizado resultante (\(S_{Res}\)) se
interpretará como aproximación observada de la agencia oscura residual a
\(G\). \(S_{Res}\) se incorporará como variable observada en un modelo
de rutas en la Fase 2.
\end{quote}

Para el resto de constructos (VEE, EIB, CWB-O, CWB-I, POPS, PsyCap) se
estimarán modelos CFA separados y, en su caso, un modelo confirmatorio
conjunto, evaluando fiabilidad, cargas factoriales y validez
discriminante (por ejemplo, comparando modelos con correlaciones
latentes libres versus restringidas a 1 y examinando cambios en CFI y
BIC).

Finalmente, se evaluará de manera específica la validez discriminante
entre VEE y EIB, dada la proximidad conceptual de ambos constructos. Se
considerará evidencia suficiente una correlación latente claramente
inferior a .80 y mejor ajuste de los modelos que preservan la distinción
entre ambos factores.

\textbf{Fase 1b. Evaluación de varianza del método común (CMV)}

La varianza del método común se abordará mediante una estrategia
principal basada en variable marcadora. Se incluirá en el cuestionario
una variable marcadora conceptualmente irrelevante para el modelo
teórico, pero con el mismo formato de respuesta (Williams et al., 2010).
En un modelo CFA específico se compararán soluciones con y sin la
inclusión de la covarianza de la variable marcadora con los demás
indicadores, siguiendo la técnica de covarianza parcial. Si los
parámetros sustantivos se mantienen estables, se considerará que la CMV
no distorsiona de manera crítica las relaciones estimadas.

Adicionalmente, como análisis de sensibilidad en un modelo reducido, se
podrá especificar un factor de método latente no medido (ULMF) al que
cargarán todos los ítems, manteniendo ortogonalidad con los factores
sustantivos. Este modelo no formará parte del análisis estructural
principal y se utilizará únicamente para comprobar la robustez de los
resultados frente a la CMV (Podsakoff et al., 2003; Williams et al.,
2010).

\textbf{Fase 2. Modelo estructural principal}

En la segunda fase se testearán las hipótesis H1--H4 mediante modelos de
ecuaciones estructurales (Escenario A) o análisis de rutas con variables
observadas (Escenario B). En ambos casos se utilizará bootstrapping (5
000 remuestreos) para estimar intervalos de confianza de los efectos
indirectos y de las interacciones.

\begin{itemize}
\item
  En el Escenario A, \(G\) y \(S_{Agencia}\) se modelarán como factores
  latentes, junto con VEE, POPS, PsyCap, EIB, CWB-O y CWB-I. El modelo
  estructural principal incluirá:

  \begin{itemize}
  \item
    Efectos directos de \(G\) y \(S_{Agencia}\) sobre EIB, CWB-O y CWB-I
    (H1).
  \item
    Efecto indirecto de \(S_{Agencia}\) sobre EIB a través de VEE (H2),
    interpretado como mediación disposicional concurrente.
  \item
    Interacción latente \(S_{Agencia} \times POPS\) en la ecuación que
    predice VEE (H3), estimada mediante integración numérica (por
    ejemplo, enfoque LMS; Marsh et al., 2004; Muthén \& Asparouhov,
    2006).
  \item
    PsyCap como predictor directo de EIB y, en análisis complementario,
    como moderador \(S_{Agencia} \times PsyCap\) sobre EIB (H4
    complementaria).
  \end{itemize}
\item
  En el Escenario B, se utilizarán \(G\) (índice observado) y
  \(S_{Res}\) como predictores en un modelo de rutas con variables
  observadas. Las interacciones se modelarán mediante términos producto
  centrados (por ejemplo, \(S_{Res} \times POPS\) para la ecuación de
  VEE; \(S_{Res} \times PsyCap\) en análisis complementarios). La
  estructura de hipótesis se mantendrá análoga al Escenario A, adaptada
  al nivel observado.
\end{itemize}

Las interacciones significativas se ilustrarán mediante gráficas de
pendientes simples y pruebas de diferencias entre pendientes en niveles
altos y bajos de los moderadores (por ejemplo, ±1 desviación estándar),
siguiendo recomendaciones estándar para la interpretación de efectos de
moderación (Aiken \& West, 1991; Hayes, 2018).

En todos los modelos se controlarán covariables demográficas básicas
(por ejemplo, edad, género, sector) cuando resulte pertinente. Dado el
tamaño muestral y la complejidad del modelo, se limitará el número de
interacciones simultáneas a aquellas que correspondan a las hipótesis
focales (POPS como moderador principal; PsyCap solo en análisis
complementarios), evitando especificaciones sobrecargadas que reduzcan
la estabilidad de las estimaciones (Wolf et al., 2013).

\textbf{Fase 3. Análisis de perfiles latentes (LPA)}

En la tercera fase se realizará un Análisis de Perfiles Latentes (LPA)
de carácter exploratorio (H5), utilizando como indicadores \(G\),
\(S_{Agencia}\) (o \(S_{Res}\), según el escenario) y los tres
\emph{outcomes} del estudio (EIB, CWB-O, CWB-I). Se explorarán
soluciones con entre dos y cuatro clases, y la elección del número de
perfiles se basará en criterios de información (BIC), el test de razón
de verosimilitud ajustado de Lo-Mendell-Rubin, la entropía y, sobre
todo, la interpretabilidad teórica (Nylund et al., 2007).

Se utilizarán múltiples conjuntos de valores iniciales aleatorios para
reducir el riesgo de convergencia a máximos locales, y se verificará la
estabilidad de las soluciones entre corridas. Se prestará especial
atención al tamaño y la estabilidad de las clases minoritarias. Perfiles
con proporciones muy reducidas o alta inestabilidad entre soluciones
cercanas se interpretarán con extrema cautela. La eventual
identificación de un perfil compatible con la noción de ``desviación
constructiva orientada al logro'' se considerará evidencia exploratoria
y descriptiva, y no constituirá requisito para sostener la validez del
modelo estructural principal ni se utilizará para afirmaciones causales.

\textbf{3.6. Consideraciones éticas}

El estudio se desarrollará conforme a las normas éticas de investigación
con seres humanos:

\begin{itemize}
\item
  aprobación previa por el comité de ética de la Universidad Nacional
  Mayor de San Marcos,
\item
  consentimiento informado explícito,
\item
  anonimato y tratamiento confidencial de los datos,
\item
  eliminación de cualquier información identificable, y
\item
  comunicación agregada de los resultados.
\end{itemize}

Dada la naturaleza sensible de las preguntas sobre rasgos oscuros y
conductas contraproducentes, se enfatizará que no se generarán reportes
individuales para empleadores y que los datos se usarán únicamente con
fines de investigación académica. Asimismo, se indicará de manera
explícita que la participación no tendrá consecuencias laborales y que
no se compartirá información individual con ninguna organización
colaboradora.

REFERENCIAS.

Aiken, L. S., \& West, S. G. (1991). \emph{Multiple regression: Testing
and interpreting interactions}. Sage.

Afshar Jahanshahi, A., Mendoza, M. I. R., \& Vicuña, J. C. A. (2023).
Towards sustainable businesses in Latin America: The role of worker's
Dark Triad personality traits. \emph{Sustainable Development},
\emph{31}(5), 3196--3206. \url{https://doi.org/10.1002/sd.2574}

Afshar Jahanshahi, A., Schmitt, V. G. H., Rivas-Mendoza, M. I.,
Fernandez-Telleria, B., da Costa, P. R., García, X. C., Ibarra, V. G.,
Nuñez, J. G., Carbonell, S. T., García, F. I., Izaguirre, L. A. P.,
Leyva, A. O., Pigola, A., \& Galera, V. (2025). How `dark' are Latino:
Implications for nascent entrepreneurship. \emph{Personality and
Individual Differences}, \emph{233}, 112897.
\url{https://doi.org/10.1016/j.paid.2024.112897}

Ali, F., \& Chamorro-Premuzic, T. (2010). The dark side of love and life
satisfaction: Associations with intimate relationships, psychopathy and
Machiavellianism. \emph{Personality and Individual Differences},
\emph{48}(2), 228--233. \url{https://doi.org/10.1016/j.paid.2009.10.016}

Ashford, S. J., \& Black, J. S. (1996). Proactivity during
organizational entry: The role of desire for control. \emph{Journal of
Applied Psychology}, \emph{81}(2), 199--214.
\url{https://doi.org/10.1037/0021-9010.81.2.199}

Atinc, G., Darrat, M., Fuller, B., \& Parker, B. W. (2010). Perceptions
of organizational politics: A meta-analysis of theoretical antecedents.
\emph{Journal of Management, 36}(4), 494--522.

Avey, J. B., Reichard, R. J., Luthans, F., \& Mhatre, K. H. (2011).
Meta‐analysis of the impact of positive psychological capital on
employee attitudes, behaviors, and performance. \emph{Human Resource
Development Quarterly}, \emph{22}(2), 127--152.
\url{https://doi.org/10.1002/hrdq.20070}

Bader, M., Hartung, J., Hilbig, B. E., Zettler, I., Moshagen, M., \&
Wilhelm, O. (2021). Themes of the dark core of personality.
\emph{Psychological Assessment, 33}(6), 511--525.
\url{https://doi.org/10.1037/pas0001006}

Baker, T., \& Nelson, R. E. (2005). Creating Something from Nothing:
Resource Construction through Entrepreneurial Bricolage.
\emph{Administrative Science Quarterly}, \emph{50}(3), 329--366.
\url{https://doi.org/10.2189/asqu.2005.50.3.329}

Baron, R. A., Franklin, R. J., \& Hmieleski, K. M. (2016). Why
Entrepreneurs Often Experience \emph{Low} , Not High, Levels of Stress.
\emph{Journal of Management}, \emph{42}(3), 742--768.
\url{https://doi.org/10.1177/0149206313495411}

Baumol, W. J. (1990). Entrepreneurship: Productive, Unproductive, and
Destructive. \emph{Journal of Political Economy}, \emph{98}(5, Part 1),
893--921. \url{https://doi.org/10.1086/261712}

Bennett, R. J., \& Robinson, S. L. (2000). Development of a measure of
workplace deviance. \emph{Journal of Applied Psychology}, \emph{85}(3),
349--360. \url{https://doi.org/10.1037/0021-9010.85.3.349}

Berisha, G., Lajçi, R., Caputo, A., \& Krasniqi, B. A. (2025). Enhancing
managerial intrapreneurship: investigating the~influence of
psychological capital, well-being and work--life balance.
\emph{International Journal of Entrepreneurial Behavior \& Research},
\emph{31}(11), 336--358.
\url{https://doi.org/10.1108/IJEBR-02-2024-0095}

Boman, B. (2024). The Gray Nine and parallel personality patterns: Big
Five, Dark Tetrad, and a ``well-rounded personality''. \emph{Integrative
Psychological \& Behavioral Science, 58}(4), 1300--1316.
\url{https://doi.org/10.1007/s12124-024-09842-y}

Bornovalova, M. A., Choate, A. M., Fatimah, H., Petersen, K. J., \&
Wiernik, B. M. (2020). Appropriate use of bifactor analysis in
psychopathology research: Appreciating benefits and limitations.
\emph{Biological Psychiatry, 88}(1), 18--27.
\url{https://doi.org/10.1016/j.biopsych.2020.01.013}

Buckels, E. E., Jones, D. N., \& Paulhus, D. L. (2013). Behavioral
Confirmation of Everyday Sadism. \emph{Psychological Science},
\emph{24}(11), 2201--2209.
\url{https://doi.org/10.1177/0956797613490749}

Carless, S. A., Wearing, A. J., \& Mann, L. (2000). A Short Measure of
Transformational Leadership. \emph{Journal of Business and Psychology},
\emph{14}(3), 389--405. \url{https://doi.org/10.1023/A:1022991115523}

Cavero, O. (2017). \emph{El trabajo en una economía heterogénea y
marginal: Un panorama general de la situación socio-económica de los
trabajadores en el Perú}. En O. Manky (Ed.), \emph{Trabajo y sociedad.
Estudios sobre el mundo del trabajo en el Perú} (pp. 26--57). Lima:
CISEPA-PUCP.

Chatterjee, A., \& Hambrick, D. C. (2007). It's All about Me:
Narcissistic Chief Executive Officers and Their Effects on Company
Strategy and Performance. \emph{Administrative Science Quarterly},
\emph{52}(3), 351--386. \url{https://doi.org/10.2189/asqu.52.3.351}

Crawford, J. J., Smyth, C., Crossey, B. P., \& Waldeck, D. (2025). The
Dark Five: A reconsideration of the Short Dark Tetrad (SD4).
\emph{Personality and Individual Differences}, \emph{235}, 112971.
\url{https://doi.org/10.1016/j.paid.2024.112971}

Dahling, J. J., Whitaker, B. G., \& Levy, P. E. (2009). The Development
and Validation of a New Machiavellianism Scale. \emph{Journal of
Management}, \emph{35}(2), 219--257.
\url{https://doi.org/10.1177/0149206308318618}

Dang-Van, T., Vo-Thanh, T., Usman, M., \& Nguyen, N. (2022).
Investigating employees' deviant work behavior in the hotel industry
during COVID-19: Empirical evidence from an emerging country.
\emph{Tourism Management Perspectives}, \emph{44}, 101042.
\url{https://doi.org/10.1016/j.tmp.2022.101042}

Douglas, H. (2000). Inductive Risk and Values in Science.
\emph{Philosophy of Science}, \emph{67}(4), 559--579.
\url{https://doi.org/10.1086/392855}

Dulović, A., Smyth, C., \& Waldeck, D. (2026). The Dark Five -
statistical anomaly or persistent phenomena: An exploratory
investigation into the factor structure of the SD4 in multiple samples.
\emph{Personality and Individual Differences}, \emph{250}, 113529.
\url{https://doi.org/10.1016/j.paid.2025.113529}

Eid, M., Geiser, C., Koch, T., \& Heene, M. (2017). Anomalous results in
G-factor models: Explanations and alternatives. \emph{Psychological
Methods}, \emph{22}(3), 541--562.
\url{https://doi.org/10.1037/met0000083}

Francke, P. (2022, agosto). \emph{Pobreza, exclusión y desigualdades en
el Perú}. Presentación, IEEP--PUCP.

Gamache, D., Maheux-Caron, V., Théberge, D., Côté, A., Rancourt, M.-A.,
Hétu, S., \& Savard, C. (2023). Revisiting the vulnerable dark triad
hypothesis using a bifactor model. \emph{Scandinavian Journal of
Psychology, 64}(5), 679--692. \url{https://doi.org/10.1111/sjop.12921}

Gardiner, E., \& Jackson, C. J. (2015). Personality and learning
processes underlying maverickism. \emph{Journal of Managerial
Psychology}, \emph{30}(6), 726--740.
\url{https://doi.org/10.1108/JMP-07-2012-0230}

Gawke, J. C., Gorgievski, M. J., \& Bakker, A. B. (2017). Employee
intrapreneurship and work engagement: A latent change score approach.
\emph{Journal of Vocational Behavior}, \emph{100}, 88--100.
\url{https://doi.org/10.1016/j.jvb.2017.03.002}

Gawke, J. C., Gorgievski, M. J., \& Bakker, A. B. (2019). Measuring
intrapreneurship at the individual level: Development and validation of
the Employee Intrapreneurship Scale (EIS). \emph{European Management
Journal}, \emph{37}(6), 806--817.
\url{https://doi.org/10.1016/j.emj.2019.03.001}

Gojny-Zbierowska, M. (2024). When there is no justice, we need an old
HERO. The trickle-down effect of psychological capital: the moderating
role of organizational justice and leaders' age. \emph{Frontiers in
Psychology}, \emph{15}. \url{https://doi.org/10.3389/fpsyg.2024.1256721}

Guerrero-Alcedo, W. G., \& Espina-Romero, H. (2024). Psychometric
properties of the Psychological Capital Questionnaire Short Version
(PCQ-12) in Peruvian university students: A Bayesian structural equation
modeling approach. \emph{Current Psychology.} Advance online
publication. \url{https://doi.org/10.1007/s12144-024-05858-6}

Haas, B. W., Campbell, W. K., Lou, X., \& Xia, R. J. (2025). All You
Nonconformists Are (Not) All Alike: Dissociable Social Stereotypes of
Mavericks and Contrarians. \emph{Personality and Social Psychology
Bulletin}, \emph{51}(10), 1847--1864.
\url{https://doi.org/10.1177/01461672231217630}

Han, J., Kamber, M., \& Pei, J. (2012). \emph{Data mining: Concepts and
techniques} (3rd ed.). Morgan Kaufmann.

Harms, P. D., Marbut, A., Johnston, A. C., Lester, P., \& Fezzey, T.
(2022). Exposing the darkness within: A review of dark personality
traits, models, and measures and their relationship to insider threats.
\emph{Journal of Information Security and Applications}, \emph{71},
103378. \url{https://doi.org/10.1016/j.jisa.2022.103378}

Hayes, A. F. (2018). \emph{Introduction to mediation, moderation, and
conditional process analysis: A regression-based approach} (2nd ed.).
Guilford Press.

Hilbig, B. E., Thielmann, I., Klein, S. A., Moshagen, M., \& Zettler, I.
(2021). The dark core of personality and socially aversive
psychopathology. \emph{Journal of Personality, 89}(2), 216--227.
\url{https://doi.org/10.1111/jopy.12577}

Hilbig, B. E., \& Moshagen, M. (2025). Four big problems of big five
agreeableness. \emph{Current Opinion in Psychology}, \emph{65}, 102086.
\url{https://doi.org/10.1016/j.copsyc.2025.102086}

Hochwarter, W. A., Kacmar, C., Perrewé, P. L., \& Johnson, D. (2003).
Perceived organizational support as a mediator of the relationship
between politics perceptions and work outcomes. \emph{Journal of
Vocational Behavior}, \emph{63}(3), 438--456.
\url{https://doi.org/10.1016/S0001-8791(02)00048-9}

Hobfoll, S. E. (2001). The Influence of Culture, Community, and the
Nested‐Self in the Stress Process: Advancing Conservation of Resources
Theory. \emph{Applied Psychology}, \emph{50}(3), 337--421.
\url{https://doi.org/10.1111/1464-0597.00062}

Horsten, L. K., Moshagen, M., Zettler, I., \& Hilbig, B. E. (2021).
Theoretical and empirical dissociations between the Dark Factor of
Personality and low Honesty-Humility. \emph{Journal of Research in
Personality}, \emph{95}, 104154.
\url{https://doi.org/10.1016/j.jrp.2021.104154}

Horsten, L. K., Thielmann, I., Moshagen, M., Zettler, I., Scholz, D., \&
Hilbig, B. E. (2024). Testing the equivalence of the aversive core of
personality and a blend of agreeableness(‐related) items. \emph{Journal
of Personality}, \emph{92}(2), 393--404.
\url{https://doi.org/10.1111/jopy.12830}

Instituto Nacional de Estadística e Informática. (2024).
\emph{{[}Informe sobre informalidad laboral en el sector servicios en
Lima Metropolitana{]}.} INEI.

Jiménez, F. (2010). \emph{La economía peruana del último medio siglo:
Ensayos de interpretación}. Lima: Departamento de Economía -- CISEPA,
Pontificia Universidad Católica del Perú.

Johnson, R. A., \& Wichern, D. W. (2007). \emph{Applied multivariate
statistical analysis} (6th ed.). Pearson Prentice Hall.

Jonason, P. K., \& Webster, G. D. (2010). The dirty dozen: A concise
measure of the dark triad. \emph{Psychological Assessment},
\emph{22}(2), 420--432. \url{https://doi.org/10.1037/a0019265}

Jones, D. N., \& Paulhus, D. L. (2009). Machiavellianism. En M. R. Leary
\& R. H. Hoyle (Eds.), \emph{Handbook of individual differences in
social behavior} (pp. 93--108). Guilford Press.

Jordan, R. (2019). \emph{Maverickism: round pegs in square holes - the
who, what and why of mavericks in organisations}.
\url{https://doi.org/10.14264/uql.2019.795}

Jordan, R., Fitzsimmons, T. W., \& Callan, V. J. (2021). Fear Not Your
Mavericks! Their Bounded Non-conformity and Positive Deviance Helps
Organizations Drive Change and Innovation. In \emph{Strategic Responses
for a Sustainable Future: New Research in International Management} (pp.
123--146). Emerald Publishing Limited.
\url{https://doi.org/10.1108/978-1-80071-929-320214006}

Jordan, R., Fitzsimmons, T. W., \& Callan, V. J. (2023). Positively
Deviant: New Evidence for the Beneficial Capital of Maverickism to
Organizations. \emph{Group \& Organization Management}, \emph{48}(5),
1254--1305. \url{https://doi.org/10.1177/10596011221102297}

Jordan, R., Fitzsimmons, T. W., \& Callan, V. J. (2025). The maverick
factor: the role of positively deviant change agents in~radical
organizational change. \emph{Journal of Organizational Change
Management}, \emph{38}(8), 30--49.
\url{https://doi.org/10.1108/JOCM-05-2024-0241}

Judge, T. A., \& Zapata, C. P. (2015). The Person--Situation Debate
Revisited: Effect of Situation Strength and Trait Activation on the
Validity of the Big Five Personality Traits in Predicting Job
Performance. \emph{Academy of Management Journal}, \emph{58}(4),
1149--1179. \url{https://doi.org/10.5465/amj.2010.0837}

Kacmar, K. M., \& Carlson, D. S. (1997). Further validation of the
Perceptions of Organizational Politics Scale (POPS): A multiple sample
investigation. \emph{Journal of Management, 23}(5), 627--658.
\url{https://doi.org/10.1016/S0149-2063(97)90010-7}

Kowalski, C. M., Vernon, P. A., \& Schermer, J. A. (2021). The Dark
Triad and facets of personality. \emph{Current Psychology},
\emph{40}(11), 5547--5558.
\url{https://doi.org/10.1007/s12144-019-00518-0}

León-Pérez, J. M., Antino, M., \& León-Rubio, J. M. (2016). Adaptación
al español de la versión reducida del Cuestionario de Capital
Psicológico (PCQ-12) en una muestra de trabajadores españoles.
\emph{Revista de Psicología Social, 31}(2), 333--361.
\url{https://doi.org/10.1080/02134748.2016.1248024}

Lerner, D. A. (2016). Behavioral disinhibition and nascent venturing:
Relevance and initial effects on potential resource providers.
\emph{Journal of Business Venturing}, \emph{31}(2), 234--252.
\url{https://doi.org/10.1016/j.jbusvent.2015.11.001}

Luo, Y. (2022). A general theory of institutional complexity.
\emph{Journal of International Business Studies, 53}(5), 886--913.
\url{https://doi.org/10.1057/s41267-021-00488-x}

Luthans, F., Avolio, B. J., Avey, J. B., \& Norman, S. M. (2007).
Positive psychological capital: Measurement and relationship with
performance and satisfaction. \emph{Personnel Psychology, 60}(3),
541--572. \url{https://doi.org/10.1111/j.1744-6570.2007.00083.x}

Ma, G. X., Born, M. P., Petrou, P., \& Bakker, A. B. (2021). Bright
sides of dark personality? A cross‐cultural study on the dark triad and
work outcomes. \emph{International Journal of Selection and Assessment},
\emph{29}(3--4), 510--518. \url{https://doi.org/10.1111/ijsa.12342}

Mair, J., \& Martí, I. (2009). Entrepreneurship in and around
institutional voids: A case study from Bangladesh. \emph{Journal of
Business Venturing, 24}(5), 419--435.
\url{https://doi.org/10.1016/j.jbusvent.2008.04.006}

Malisetty, S., \& Kumari, K. V. (2018). Exploring entrepreneurship as
deviant behaviour. \emph{International Journal of Entrepreneurial
Behavior \& Research, 24}(3), 644--662.

Mardia, K. V., Kent, J. T., \& Bibby, J. M. (1979). \emph{Multivariate
analysis}. Academic Press.

Marsh, H. W., Wen, Z., \& Hau, K.-T. (2004). Structural Equation Models
of Latent Interactions: Evaluation of Alternative Estimation Strategies
and Indicator Construction. \emph{Psychological Methods}, \emph{9}(3),
275--300. \url{https://doi.org/10.1037/1082-989X.9.3.275}

Merani, A. L. (n.d.). Paradojas del análisis factorial y de la
cibernética. En \emph{La dialéctica en la psicología}.

Miller, J. D. (2015). \emph{Examining the role of narcissism in
leadership and organizational outcomes.} En V. Zeigler-Hill, L. L. M.
Welling, \& T. K. Shackelford (Eds.), \emph{Evolutionary perspectives on
social psychology} (pp. 321--335). Springer.

Milosevic, I., Bass, A. E., \& Milosevic, D. (2017). Leveraging Positive
Psychological Capital (PsyCap) in Crisis: A Multiphase Framework.
\emph{Organization Management Journal}, \emph{14}(3), 127--146.
https://doi.org/10.1080/15416518.2017.1353898

Moshagen, M., Hilbig, B. E., \& Zettler, I. (2018). The dark core of
personality. \emph{Psychological Review, 125}(5), 656--688.
\url{https://doi.org/10.1037/rev0000111}

Muthen, B., \& Asparouhov, T. (2006). Item response mixture modeling:
Application to tobacco dependence criteria. \emph{Addictive Behaviors},
\emph{31}(6), 1050--1066.
\url{https://doi.org/10.1016/j.addbeh.2006.03.026}

Nylund, K. L., Asparouhov, T., \& Muthén, B. O. (2007). Deciding on the
number of classes in latent class analysis and growth mixture modeling:
A Monte Carlo simulation study. \emph{Structural Equation Modeling,
14}(4), 535--569. \url{https://doi.org/10.1080/10705510701575396}

Paulhus, D. L., \& Williams, K. M. (2002). The Dark Triad of
personality: Narcissism, Machiavellianism, and psychopathy.
\emph{Journal of Research in Personality, 36}(6), 556--563.
\url{https://doi.org/10.1016/S0092-6566(02)00505-6}

Paulhus, D. L., Buckels, E. E., Trapnell, P. D., \& Jones, D. N. (2021).
Development and validation of the Short Dark Tetrad (SD4).
\emph{Assessment, 28}(1), 234--259.
\url{https://doi.org/10.1177/1073191120922611}

Pinto, A. (1970). Naturaleza e implicancias de la ``heterogeneidad
estructural'' de la economía latinoamericana. \emph{El Trimestre
Económico, 37}(145), 83--100.

Pinto, A., \& Di Filippo, A. (1979). Desarrollo y pobreza en la América
Latina: un enfoque histórico--estructural. \emph{El Trimestre Económico,
183}(3), 569--590.

Pletzer, J. L. (2021). Why do dark personalities predict
counterproductive work behavior? A meta-analysis of indirect effects.
\emph{European Journal of Work and Organizational Psychology, 30}(6),
868--888. \url{https://doi.org/10.1080/1359432X.2021.1919955}

Podsakoff, P. M., MacKenzie, S. B., Lee, J. Y., \& Podsakoff, N. P.
(2003). Common method biases in behavioral research: A critical review
of the literature and recommended remedies. \emph{Journal of Applied
Psychology, 88}(5), 879--903.
\url{https://doi.org/10.1037/0021-9010.88.5.879}

Ramírez-Vilca, S. (2022). \emph{{[}Tesis de maestría sobre clima ético,
percepción de política organizacional y conducta desviada en
trabajadores de Lima Metropolitana{]}}. Tesis de maestría no publicada,
Universidad ESAN, Lima, Perú.

Ramos-Vera, C., Calle, D., Calizaya-Milla, Y. E., \& Saintila, J.
(2023). Network Analysis of Dark Triad Traits and Emotional Intelligence
in Peruvian Adults. \emph{Psychology Research and Behavior Management},
\emph{Volume 16}, 4043--4056. \url{https://doi.org/10.2147/PRBM.S417541}

Ramos-Vera, C., O'Diana, A. G., Villena, A. S., Bonfá-Araujo, B.,
Barros, L. de O., Noronha, A. P. P., Gómez-Acosta, A., Sierra-Barón, W.,
Gerymski, R., Ogundokun, R. O., Babatunde, A. N., Abdulahi, A. T., \&
Adeniyi, E. A. (2023). Dark and Light Triad: A cross-cultural comparison
of network analysis in 5 countries. \emph{Personality and Individual
Differences}, \emph{215}, 112377.
\url{https://doi.org/10.1016/j.paid.2023.112377}

Rauthmann, J. F., \& Kuper, N. (2025). The landscape of personality
psychology in the new millennium: A systematic keyword analysis of
journal articles from 2000 to 2021. \emph{Journal of Personality and
Social Psychology}, \emph{128}(3), 722--763.
\url{https://doi.org/10.1037/pspp0000532}

Robinson, S. L., \& Bennett, R. J. (1995). A typology of deviant
workplace behaviors: A multidimensional scaling study. \emph{Academy of
Management Journal, 38}(2), 555--572.
\url{https://doi.org/10.5465/256693}

Rodriguez, A., Reise, S. P., \& Haviland, M. G. (2016). Evaluating
bifactor models: Calculating and interpreting statistical indices.
\emph{Psychological Methods}, \emph{21}(2), 137--150.
\url{https://doi.org/10.1037/met0000045}

Rose, L., Sleep, C. E., Lynam, D. R., \& Miller, J. D. (2023). Welcome
to the Jangle: Comparing the Empirical Profiles of the ``Dark'' Factor
and Antagonism. \emph{Assessment}, \emph{30}(8), 2626--2643.
\url{https://doi.org/10.1177/10731911221124847}

Schreiber, A., \& Marcus, B. (2020). The place of the ``Dark Triad'' in
general models of personality: Some meta-analytic clarification.
\emph{Psychological Bulletin}, \emph{146}(11), 1021--1041.
\url{https://doi.org/10.1037/bul0000299}

Sharpe, B. M., Collison, K. L., Lynam, D. R., \& Miller, J. D. (2021).
Does Machiavellianism meaningfully differ from psychopathy? It depends.
\emph{Behavioral Sciences \& the Law}, \emph{39}(5), 663--677.
\url{https://doi.org/10.1002/bsl.2538}

Spector, P. E. (2006). Method variance in organizational research: Truth
or urban legend? \emph{Organizational Research Methods, 9}(2), 221--232.
\url{https://doi.org/10.1177/1094428105284955}

Kedrov, M., \& Spirkin, A. (1967). \emph{La ciencia}.

Tang, J., Kacmar, K. M., \& Busenitz, L. (2012). Entrepreneurial
alertness in the pursuit of new opportunities. \emph{Entrepreneurship
Theory and Practice, 36}(5), 1053--1079.
\url{https://doi.org/10.1111/j.1540-6520.2011.00452.x}

Tang, J.-J. (2020). Psychological Capital and Entrepreneurship
Sustainability. \emph{Frontiers in Psychology}, \emph{11}.
\url{https://doi.org/10.3389/fpsyg.2020.00866}

Tett, R. P., \& Burnett, D. D. (2003). A personality trait--based
interactionist model of job performance. \emph{Journal of Applied
Psychology, 88}(3), 500--517.
\url{https://doi.org/10.1037/0021-9010.88.3.500}

Torgo, L. (2017). \emph{Data mining with R: Learning with case studies}
(2nd ed.). Chapman \& Hall/CRC.

Transparency International. (2024). \emph{Corruption Perceptions Index
2024.} Transparency International.

Victor, B., \& Cullen, J. B. (1988). The Organizational Bases of Ethical
Work Climates. \emph{Administrative Science Quarterly}, \emph{33}(1),
101. \url{https://doi.org/10.2307/2392857}

Vize, C. E., Collison, K. L., Miller, J. D., \& Lynam, D. R. (2020). The
``core'' of the dark triad: A test of competing hypotheses.
\emph{Personality Disorders: Theory, Research, and Treatment, 11}(2),
91--99. \url{https://doi.org/10.1037/per0000386}

Vize, C. E., \& Lynam, D. R. (2021). On the importance of the assessment
and conceptualization of Agreeableness: A commentary on ``Agreeableness
and the common core of dark traits are functionally different
constructs.'' \emph{Journal of Research in Personality}, \emph{90},
104059. \url{https://doi.org/10.1016/j.jrp.2020.104059}

Webb, J. W., Khoury, T. A., \& Hitt, M. A. (2013). The influence of
formal and informal institutional voids on entrepreneurship.
\emph{Journal of Business Venturing, 28}(5), 598--614.
\url{https://doi.org/10.1016/j.jbusvent.2012.05.003}

Webster, G. D., Blötner, C., Wongsomboon, V., Rodriguez-Boerwinkle, R.,
\& Silvia, P. J. (2025). \emph{The Hateful Eight: An Efficient
Multifaceted Short Dark Tetrad Measure}.
\url{https://doi.org/10.31234/osf.io/pr4u6_v4}

Welter, F., \& Smallbone, D. (2011). Institutional perspectives on
entrepreneurial behavior in challenging environments. \emph{Journal of
Small Business Management, 49}(1), 107--125.
\url{https://doi.org/10.1111/j.1540-627X.2010.00317.x}

Williams, L. J., Hartman, N., \& Cavazotte, F. (2010). Method variance
and marker variables: A review and comprehensive CFA marker technique.
\emph{Organizational Research Methods, 13}(3), 477--514.
\url{https://doi.org/10.1177/1094428110366036}

Wolf, E. J., Harrington, K. M., Clark, S. L., \& Miller, M. W. (2013).
Sample size requirements for structural equation models: An evaluation
of power, bias, and solution propriety. \emph{Educational and
Psychological Measurement, 73}(6), 913--934.
\url{https://doi.org/10.1177/0013164413495237}

World Justice Project. (2024). \emph{World Justice Project Rule of Law
Index 2024.} World Justice Project. URL del informe.

Zegarra-López, A. C., Uribe-Bravo, K. A., Berrocal-Aragonés, G.,
Prieto-Molinari, D. E., \& López, M. C. (2024). Adaptation and
psychometric properties of the Short Dark Tetrad in a Peruvian
sample.~\emph{TPM: Testing, Psychometrics, Methodology in Applied
Psychology},~\emph{31}(2). \url{https://doi.org/10.4473/TPM31.2.3}
